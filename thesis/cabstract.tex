\chapter*{摘~~要}
\addcontentsline{toc}{chapter}{中文摘要}

%基本資訊

\noindent
論文名稱:跨平台軟體持續整合樣式之已知案例探討			\hspace*{\fill}頁數:47		            \\
校所別:國立台北科技大學~資訊工程系碩士班					  		  						\\
畢業時間:九十九學年度第二學期					\hspace*{\fill}學位:碩士		        \\
研究生:王熙鈞								\hspace*{\fill}指導教授:鄭有進、謝金雲教授\\
\hspace*{\fill}\\
\noindent
關鍵字:跨平台軟體開發,樣式,樣式語言,持續整合\\
\hspace*{\fill}\\
%
\indent

軟體開發團隊實踐持續整合,藉此順利完成軟體開發,這是軟體開發業界逐漸採用的作法。一如軟體開發中眾多問題的解決方法可用樣式加以表達,持續整合的作法亦適宜以樣式表達之。關於持續整合,目前已有若干持續整合樣式及樣式語言被整理提出,用以協助軟體開發團隊在開發跨平台軟體時實踐持續整合。雖然這些跨平台軟體持續整合樣式語言是以特定應用領域的跨平台軟體專案實踐持續整合的經驗為基礎,但要真正成為一個樣式語言,其中之樣式需要有更多的實例。因此本論文以開放原始碼的跨平台軟體專案\textendash\hspace{4pt}Chromium與SWT,作為案例探討的對象,在其中尋找已知案例。我們在實例中已找到了九個樣式的已知案例,足以說明跨平台軟體持續整合樣式語言的適用性。此外,本論文亦藉由於實例中所得到實踐持續整合的經驗與知識,輔以他人已經發表關於持續整合的樣式,整理適當的樣式融入跨平台軟體持續整合樣式語言中,進一步演化此一樣式語言。

