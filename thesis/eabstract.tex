\chapter*{ABSTRACT}
\addcontentsline{toc}{chapter}{ABSTRACT}

%基本資訊

\noindent
Title:On the known use of a collection of Patterns for Continuous Integration in Cross-Platform Software Development\hspace*{\fill}Pages:47		   \\
School:National Taipei University of Technology						  		   \\
Department:Computer Science and Information Engineering						   \\
Time:June,2011									  \hspace*{\fill}Degree:Master	   \\
Researcher:Si-Jyun Wang							  \hspace*{\fill}Advisor:C.-Y. Hsieh and Y.C.Cheng\\
\hspace*{\fill}\\
Keywords:Cross-Platform Software Development,Pattern,Pattern Language,Continuous Integration\\
\hspace*{\fill}\\
%
\indent
%In this thesis. Taking open source cross-platform software projects as study cases, to discover known uses of the Continuous Integration Patterns for Cross-Platform Software Development. Learning the experiment and knownledge of practicing Continuous Integration from the developers of the study cases, and the patterns about Continuous Integration that already published. And then extracting the essentials from both to help evoluting the Continuous Integration Patterns for Cross-Platform Software Development. Finally, to make the petterns approching completed.
There are well-known success stories that software development teams practicing continuous integration produce quality software. Recently, a pattern language for continuous integration in developing cross-platform software was proposed, helping software development teams to practice continuous integration when developing cross-platform software. The discussion of the known uses of the pattern language is still lacking, however. In this thesis, taking Chromium and SWT\hspace{3pt}\textendash\hspace{3pt}open source  cross-platform projects as study cases, known uses of these patterns are found. In all, we found nine known uses of the patterns, thus proving that the pattern language is a working and living one. In addition, we also consider other known continuous integration patterns and incorporate them to further evolve the pattern language for continuous integration in developing cross-platform software.
