\chapter{結論}
本章節的第一小節描述本論文的貢獻,第二小節描述後續可能的研究方向。
\section{論文貢獻}
本論文貢獻如以下三點描述。
\begin{description}
\item[提出用以觀察跨平台軟體持續整合樣式已知案例之研究方法] 本論文提出以軟體建置、檔案目錄結構、架構設計、專案社群文件四個構面為途徑進行探討樣式已知案例之方法,實際利用該方法觀察確實觀察到樣式之已知案例。

\item[探討已知案例] 於實際案例中,針對跨平台軟體持續整合樣式語言中三大範疇樣式語言,本論文確實發現上述三大範疇樣式語言之已知案例,實證該樣式語言之有效性、真實性。請見本論文第三章。

\item[演化樣式語言] 於實際案例中,本論文歸納其開發團隊對於實踐持續整合的經驗與知識成為樣式,並且擷取他人已經發表關於持續整合的樣式。綜合上述兩者中所得知樣式,我們將其加入跨平台軟體持續整合樣式語言,藉此演化跨平台軟體持續整合樣式語言。請見本論文第四章。
\end{description}

\section{未來可能的研究方向}
未來可能的研究方向如以下四點描述。
\begin{description}
\item [樣式語言必須與時俱進] 如果其他實際跨平台軟體專案實踐持續整合的經驗,無法利用跨平台軟體持續整合樣式語言表達時,則必須進行調整或是演化該樣式語言的動作。

\item[探討其他類型之程式語言實作之已知案例] 利用直譯式程式語言撰寫的跨平台軟體,例如以Python, Ruby與Perl語言實作之跨平台軟體,是否也如同以C, C++與Java語言等編譯式程式語言撰寫的跨平台軟體一般,可以找到跨平台軟體持續整合樣式語言的已知案例。

\item[以貼近持續整合之觀點深入探討] 針對Project範疇樣式語言進行探討,本論文以架構設計的觀點進行討論。針對跨平台軟體開發團隊導入樣式後所帶來的優勢,我們並未以持續整合的觀點深入探討,後續工作需以貼近持續整合之觀點進行探討樣式之已知案例。

\item[針對Chromium專案進行實驗] 針對Chromium專案,Single Shared Library Project樣式描述子專案間可以參照其他子專案之產出而不需完全重新建置底層元件,我們將導入該樣式於Chromium專案中,期望可以縮短Chromium專案建置之總時間。針對上述概念,我們將繼續規劃相關實驗進行驗證。若進行實驗所得到正面之結果,我們可以向Chromium專案提出建議,訴說導入Single Shared Library樣式之優點與必須額外克服的缺點。
\end{description}