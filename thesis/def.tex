%定義環境變數與巨集指令
%-------------------基本設定------------------------------------%
% 沿用 latex 的一些標點的轉換,如 en-dash 以兩個減號表示
\defaultfontfeatures{Mapping=tex-text}
\XeTeXlinebreaklocale "zh"
\XeTeXlinebreakskip = 0pt plus 1pt
%-------------------------------------------------------------%

%-------------------重新定義章節名稱格式--------------------------%

\titleformat{\chapter}[hang]{\centering\huge\bfseries}{第\CJKnumber{\thechapter}章}{1em}{}
%\renewcommand\chaptername{Chapter\hspace{.5em}\thechapter}
\renewcommand\contentsname{目錄}
\renewcommand\listfigurename{圖目錄}
\renewcommand\listtablename{表目錄}

\titlecontents{chapter}[0pt]{}
    {第\CJKnumber{\thecontentslabel}章\quad}{}
    {\hspace{.5em}\titlerule*[10pt]{$\cdot$}\contentspage}
\titlecontents{section}[2em]{}
    {\thecontentslabel\quad}{}
    {\hspace{.5em}\titlerule*[10pt]{$\cdot$}\contentspage}
\titlecontents{subsection}[4em]{}
    {\thecontentslabel\quad}{}
    {\hspace{.5em}\titlerule*[10pt]{$\cdot$}\contentspage}

% \renewcommand{\refname}{參考文獻}
\renewcommand{\bibname}{參考文獻}			%修改參考文獻的標題名
\titlespacing{\chapter}{0pt}{*0}{*4}		%設定標題與四周的距離
\titlelabel{\thetitle\quad}				%設定章節標題的樣式
\renewcommand{\figurename}{圖}
\renewcommand{\tablename}{表}
%-----------------------------------------------------------------%

%-------------------設定行距--------------------------------------%
\renewcommand{\baselinestretch}{1.6}
%\linespread{1.6}
%設定enumerate等的item間距

%-----------------------------------------------------------------%

%-------------------定義浮水印------------------------------------%
\newcommand\WatermarkPicture{%
   \put(0,0){%
   \parbox[b][\paperheight]{\paperwidth}{%
     \vfill
     \centering
     \includegraphics[width=5cm,keepaspectratio]{watermark_ntut.png}%
     \vfill
     }
   }
}

%-----------------------------------------------------------------%

%------------------圖片巨集---------------------------------------%

%巨集格式
%mygraphic{圖片KeyWord}{圖片註解}{圖片路徑}
\def\myGraphic#1#2#3
{
	\begin{figure}[!htbp]
		\begin{center}
			\includegraphics[width=\textwidth]{#3}
			\caption{#2}\label{#1}
		\end{center}
		
	\end{figure}
}
%-----------------------------------------------------------------%

%------------------小小圖片巨集---------------------------------------%

%巨集格式
%mygraphic{圖片KeyWord}{圖片註解}{圖片路徑}
\def\myGraphicSS#1#2#3
{
	\begin{figure}[!htbp]
		\begin{center}
			\includegraphics[width=2cm]{#3}
			\caption{#2}\label{#1}
		\end{center}
		
	\end{figure}
}
%-----------------------------------------------------------------%

%------------------小圖片巨集---------------------------------------%

%巨集格式
%mygraphic{圖片KeyWord}{圖片註解}{圖片路徑}
\def\myGraphicS#1#2#3
{
	\begin{figure}[!htbp]
		\begin{center}
			\includegraphics[width=6cm]{#3}
			\caption{#2}\label{#1}
		\end{center}
		
	\end{figure}
}
%-----------------------------------------------------------------%

%------------------中圖片巨集---------------------------------------%

%巨集格式
%mygraphic{圖片KeyWord}{圖片註解}{圖片路徑}
\def\myGraphicM#1#2#3
{
	\begin{figure}[!htbp]
		\begin{center}
			\includegraphics[width=8cm]{#3}
			\caption{#2}\label{#1}
		\end{center}
		
	\end{figure}
}
%-----------------------------------------------------------------%

%------------------大圖片巨集---------------------------------------%

%巨集格式
%mygraphic{圖片KeyWord}{圖片註解}{圖片路徑}
\def\myGraphicB#1#2#3
{
	\begin{figure}[!htbp]
		\begin{center}
			\includegraphics[width=12cm]{#3}
			\caption{#2}\label{#1}
		\end{center}
		
	\end{figure}
}
%-----------------------------------------------------------------%

%------------------大大圖片巨集---------------------------------------%

%巨集格式
%mygraphic{圖片KeyWord}{圖片註解}{圖片路徑}
\def\myGraphicBB#1#2#3
{
	\begin{figure}[!htbp]
		\begin{center}
			\includegraphics[width=16cm]{#3}
			\caption{#2}\label{#1}
		\end{center}
		
	\end{figure}
}
%-----------------------------------------------------------------%

%------------------表格巨集---------------------------------------%
\renewcommand{\arraystretch}{1}

%巨集格式
%myTable{Table KeyWord}{Table註解}
\def\myTable#1#2
{
	\begin{table}[!htbp]
	\setlength{\abovecaptionskip}{0pt}
	\setlength{\belowcaptionskip}{10pt}
	\begin{center}
	\caption{#2}\label{#1}
}


\def\endmyTable
{
	\end{center}
	\end{table}
}

%-----------------------------------------------------------------%

%------------------修改圖與表的註解編號格式-----------------------%

\makeatletter
\long\def\@makecaption#1#2{%
  \vskip\abovecaptionskip
  \sbox\@tempboxa{{\bfseries #1}\quad #2}%
  \ifdim \wd\@tempboxa >\hsize
    {\bfseries #1}\quad #2\par
  \else
    \global \@minipagefalse
    \hb@xt@\hsize{\hfil\box\@tempboxa\hfil}%
  \fi
  \vskip\belowcaptionskip}
\makeatother


%-----------------------------------------------------------------%
%------------------修改Description縮排格式--------------------------%
%\makeatletter
%\renewenvironment{description}
%  {\list{}{\leftmargin\z@ \labelwidth\z@ \itemindent-\leftmargin
%   \let\makelabel\descriptionlabel}}
%  {\endlist}
%\makeatother
%-----------------------------------------------------------------%
%------------------Use Case巨集---------------------------------------%
%巨集格式
%useCase{UseCase名稱}{UseCase圖片路徑}
\def\useCase#1#2#3
{
	\subsection{#1}
	\begin{figure}[!htb]
		\begin{center}
			\includegraphics[width=\textwidth]{#2}
		\end{center}
	\end{figure}
}
% \graphicspath{{./picture/eps/}{./picture/png/}}
\graphicspath{{.}}					%告訴Latex去這兩個目錄下找圖檔
%-----------------------------------------------------------------%
%-------------------為了方便所設定的巨集--------------------------%

%-----------------------------------------------------------------%
