% XeLaTeX can use any Mac OS X font. See the setromanfont command below.
% Input to XeLaTeX is full Unicode, so Unicode characters can be typed directly into the source.

% The next lines tell TeXShop to typeset with xelatex, and to open and save the source with Unicode encoding.

%!TEX TS-program = xelatex
%!TEX encoding = UTF-8 Unicode

\documentclass[12pt]{article}
\usepackage{geometry}                % See geometry.pdf to learn the layout options. There are lots.
\geometry{a4paper}                   % ... or a4paper or a5paper or ... 
%\geometry{landscape}                % Activate for for rotated page geometry
%\usepackage[parfill]{parskip}    % Activate to begin paragraphs with an empty line rather than an indent
\usepackage{graphicx}
\usepackage{amssymb}

% Will Robertson's fontspec.sty can be used to simplify font choices.
% To experiment, open /Applications/Font Book to examine the fonts provided on Mac OS X,
% and change "Hoefler Text" to any of these choices.

\usepackage{fontspec,xltxtra,xunicode}
\usepackage[boldfont,slantfont,CJKnumber,CJKaddspaces]{xeCJK}
\defaultfontfeatures{Mapping=tex-text}
\setromanfont[Mapping=tex-text]{Hoefler Text}
\setsansfont[Scale=MatchLowercase,Mapping=tex-text]{Gill Sans}
\setmonofont[Scale=MatchLowercase]{Andale Mono}
\setmainfont{Times New Roman}
\setCJKmainfont{BiauKai}

\title{論文修改與口試問題回覆}
\author{王熙鈞}
%\date{}                                           % Activate to display a given date or no date

\begin{document}
\maketitle

% For many users, the previous commands will be enough.
% If you want to directly input Unicode, add an Input Menu or Keyboard to the menu bar 
% using the International Panel in System Preferences.
% Unicode must be typeset using a font containing the appropriate characters.
% Remove the comment signs below for examples.

% \newfontfamily{\A}{Geeza Pro}
% \newfontfamily{\H}[Scale=0.9]{Lucida Grande}
% \newfontfamily{\J}[Scale=0.85]{Osaka}

% Here are some multilingual Unicode fonts: this is Arabic text: {\A السلام عليكم}, this is Hebrew: {\H שלום}, 
% and here's some Japanese: {\J 今日は}.
\section{論文修改}
\begin{description}
\item[修改1] 補上圖2.1的框線,用以避免與內文混淆。
\item[修改2] 修改Reference 的順序,改以參照出現順序呈現。
\item[修改3] 修改關於論文pp.7描述關於樣式關係的說明有中英文使用不一致的問題。
\item[修改4] 補上圖3.1的虛線圖示說明。
\end{description}

\section{口試問題回覆}
\begin{description}
\item[Patch Project樣式的定義?] 定義1:小規模的程式碼變更與原本的程式碼做diff比較,依照上述比較資料定義一份腳本替換對應原始碼變更之元件。於更新軟體時,使用者執行該腳本自動替換必須更新的元件。這樣可以節省更新軟體所需要之時間。定義2:某程式設計人員對於修改開源軟體專案之程式碼有興趣,但卻無法將修改的程式碼提交進入該專案中。該名程式設計人員可以將其修改過之程式碼置放於patch project目錄,並且修改建置腳本用以產出修改過之軟體產品。
\item[在chromium project中沒有發現patch project樣式,是否是一個問題?] 
為了解決開發軟體時所遭遇問題,軟體開發團隊使用樣式。對於更新chromium瀏覽器軟體,chromium專案已經將此機制實作並內建於程式中,所有的軟體更新\hspace{3pt}\textemdash\hspace{3pt}不論規模是大或小\hspace{3pt}\textemdash\hspace{3pt}都以該機制完成。所以,chromium專案沒有太強的需求去實踐patch project樣式。
\item[Feedbacks指的是什麼?如果有不負責任的整合監控者,該如何處理?] feedbacks指的是將整合的狀態於團隊成員常用的通訊媒體中呈現。或者是在團隊之共同工作的空間中,開發團隊提供一個代表整合狀態之metaphor\hspace{3pt}\textemdash\hspace{3pt}例如,一個會發出巨大聲響之警報器。\hspace{3pt}\textemdash\hspace{3pt}用以呈現整合狀態給所有成員。上述之metaphor醒目且單純表達一件事情,團隊可以利用該metaphor與同儕間壓力避免整合監控者不負責任的情況發生。
\item[上述問題會延宕整個開發流程,如何處理?] 當發生Broken Build時,所有的成員必須停下手邊的工作,這是軟體開發團隊實踐持續整合的一個必須面對的情形,也是為了確保軟體品質必須承擔之風險。不過通常我們可以將該個照成整合問題之程式碼版本忽略,回復程式碼至之前運作正常之版本以繼續工作。
\item[慨念相近的程式碼必須分類,如何分類?設計是以top-down或是bottom-up出發?]
對於完成一項功能,我們期望元件\hspace{3pt}\textemdash\hspace{3pt}賦予完整抽象化概念\hspace{3pt}\textemdash\hspace{3pt}發揮其本身能力並與其他元件互動來達成該功能。以chromium專案為例,其團隊一開始並未針對開發跨平台軟體抽象化與平台相關之View,並且使用MVC架構樣式,而是以類似bottom-up的方式,先將功能實作出來。爾後針對方便跨平台軟體開發,將其軟體架構修改至MVC架構樣式所述求之結構,如此可視以為top-down之方式修改架構。所以上述之設計機制可以稱為middle-out。我們並不能推斷以何種方式進行設計會比較合適,可以必須綜合兩種方式用以設計軟體架構。
\item[描述自動化整合的機制]
我們會涵蓋以下步驟,並以自動化執行步驟為目標。
\begin{itemize}
\item 編譯
\item 單元測試
\item 自動化end-to-end之驗收測試
\item 靜態分析
\item 產生交付給客戶之安裝檔案
\end{itemize}

\item[描述關於持續整合中關於持續的定義] 每當有團隊成員簽入程式碼並且產生一個新的reversion時,整合流程必須被啟動,並且於軟體開發生命週期中不中斷地執行該流程。一個團隊成員每天至少簽入一次程式碼,持續整合系統一天會執行多次整合流程。
\end{description}

\end{document}  