\chapter{緒論}

開發跨平台軟體是一項於技術面、商業策略上都極具挑戰的工作。軟體公司Netscape希望各個平台上的使用者皆可利用其瀏覽器軟體瀏覽網路,所以該公司訴求其瀏覽器可以在不同平台上運行,基於上述原因,於1990年代Netscape佔據80\%瀏覽器市佔率\cite{netscape1990marketshare}。為了達成軟體於不同平台上運行的目標,Netscape定義其瀏覽器軟體為跨平台軟體,並基於同一份程式碼,同時為不同平台使用者開發瀏覽器產品。由Netscape的經驗可知開發跨平台軟體的困難包含:撰寫跨平台程式碼必須留意各個平台上的特性;應該減少專屬於特定平台的原始碼;與於不同平台上執行測試將更加耗時\cite{netscapecrossplatform}。雖然Netscape最終因為軟體生命週期管理不當、微軟競逐瀏覽器市場等因素,而逐漸式微,但是該公司對於跨平台軟體開發的經驗依然值得借鏡。

在1980年代,基於不同的商業策略,雖然由微軟所開發的Word可以同時支援Windows, Mactintosh平台,但是微軟並不是以開發跨平台軟體的手段達成,而是利用p-code\cite{microsoftpcode}的特性迴避開發跨平台軟體\cite{netscapecrossplatform}。在不需做任何修改的條件下,同一份以p-code呈現的可執行檔案可以在上述兩個平台執行,在原始碼極少需要因應不同平台而有不同實作的有利條件下,微軟將原始碼編譯為p-code同時於兩個平台上佈署,代價有二:(1)Word在Mactintosh平台上的執行速度較慢。(2)Word在Macintosh平台上的GUI功能比較匱乏。由上述內容得知,微軟的Word產品並不能算是跨平台軟體,至今微軟Office系統雖然可以於Mac OS X上執行,但是該版本Office系統是由Windows平台的Office移植過去的。綜合微軟於不同平台上經營Word軟體的策略,與Netscape發展跨平台瀏覽器的經驗,足以說明跨平台軟體開發在技術面、商業策略上的困難與挑戰。

本論文僅就開發跨平台軟體的技術面議題論述。Martin Fowler提出持續整合概念,並且主張發展一個產品的過程中,無論是軟硬體,都不能將整合工作延遲至專案開發階段的末期才開始進行,而必須於專案初期就進行整合,並且在交付產品前都不能停止\cite{martinfowlerci}。基於這樣的精神,注重及早交付產品的敏捷式軟體開發方法(Agile Methods)將持續整合視為最佳實踐\cite{xpagilemethods}。在執行持續整合時,首重確實實行各種必須涵蓋於其中的步驟,例如為了維持軟體系統品質,於整合流程中,敏捷式軟體開發團隊涵蓋各種驗證軟體品質的步驟。此外相關經驗,例如執行持續整合流程時各種錯誤排除的經驗如何於團隊成員間擴散,亦為實踐持續整合必須考慮的事項。在市面上有許多持續整合的工具,從開放原始碼到商用軟體,為數眾多,然而只依賴工具並無法傳達如何實踐持續整合。軟體開發團隊若未體悟工具無法根除執行持續整合本質上的問題,且團隊未建立起積極面對整合議題的態度,則持續整合易淪為形式。

跨平台軟體開發團隊執行持續整合流程,用以降低於不同平台上進行測試的成本。於持續整合流程中必須涵蓋各種形式的自動化測試,這樣的作法將對跨平台軟體開發帶來衝擊,因為於持續整合流程中,執行過多的測試將造成開發團隊簽入原始碼的過程冗長,如此間接使得團隊成員執行持續整合的意願降低\cite{teampace}。面對上述的問題,開發團隊需能以有效途徑找到解決方法,並與其他開發團隊分享這些經驗,以幫助其他開發團隊執行持續整合。在這方面,樣式(Pattern)為一個公認有效的途徑。

樣式於特定情形、條件下用以解決於其中發生的問題,因為發展樣式的過程必須嚴謹,所以樣式合適用來傳達執行持續整合時的經驗與知識,而實踐持續整合並無法以單一樣式表達其中經驗,所以必須將其中各個經驗發展成樣式以利傳達,並且連結樣式形成樣式語言,用以描述實踐持續整合之整體經驗與知識。於知名樣式網站Portland Pattern Repository\cite{portlandcipatterns}上,十一個關於持續整合的樣式已經發表,然而其中所刊載之樣式多半非正規形式呈現,僅為各個開發者實踐持續整合時的經驗分享。

C.-Y. Hsieh等提出跨平台軟體持續整合樣式語言\cite{crossplatformcipatterns},並以正規形式呈現其中樣式,期望跨平台軟體開發團隊利用該樣式語言,順利實踐持續整合,藉此完成跨平台軟體之開發。 雖然藉由特定領域軟體開發實踐持續整合的經驗,C.-Y. Hsieh等提出的跨平台軟體持續整合樣式語言,尚需要更多的實例加以實證之。

本論文以軟體建置、檔案目錄結構、架構設計、專案社群文件四個構面為途徑探討開放原始碼專案Chromium與SWT,期望於其中得到跨平台軟體持續整合樣式之已知案例,用以確定樣式語言存在之真實性與其有效性。 此外藉由跨平台軟體開發團隊對於實踐持續整合的經驗與知識,輔以他人已經發表過持續整合樣式,將合適的樣式與經驗融入跨平台軟體持續整合樣式語言中,進一步演化此一樣式語言。

本論文分成五個章節。第一章為緒論,介紹研究背景與動機、研究目標、論文組織與架構。第二章為背景介紹,介紹關於跨平台軟體開發、持續整合、樣式語言、跨平台軟體持續整合樣式語言等知識。第三章為實例介紹,針對真實世界中的跨平台軟體專案,就跨平台軟體持續整合樣式語言的分類,分別舉出實際案例說明。第四章為演化跨平台軟體持續整合樣式語言,針對真實案例和他人已經發表的樣式,論述其融入跨平台軟體樣式語言的根據。第五章為結論,介紹本論文的貢獻與未來可能的研究方向。


