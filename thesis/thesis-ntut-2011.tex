% XeLaTeX can use any Mac OS X font. See the setromanfont command below.
% Input to XeLaTeX is full Unicode, so Unicode characters can be typed directly into the source.

% The next lines tell TeXShop to typeset with xelatex, and to open and save the source with Unicode encoding.

%!TEX TS-program = xelatex
%!TEX encoding = UTF-8 Unicode

\documentclass[12pt,oneside,a4paper]{report}
\usepackage[bookmarks=true,bookmarksopen=true,colorlinks=false,linkcolor=black,driverfallback=dvipdfm,pdfstartview=FitH]{hyperref}	%在PDF中增加Bookmaker
%\usepackage{hyperref}

\usepackage{color}
\definecolor{bisque}{rgb}{.996,.891,.755}

\usepackage[top=2.5cm,bottom=2.75cm,left=3.5cm,right=2.5cm]{geometry}				%設定頁面留白

\usepackage[boldfont,slantfont,CJKnumber,CJKspace]{xeCJK}						%xeCJK套件
\usepackage{CJKnumb}
\setmainfont{Times New Roman}														%設定英文預設字體			
\setCJKmainfont{cwTeX Q Ming Medium}                                                                 %設定中文預設字體
\setCJKsansfont{cwTeX Q Kai Medium}



\usepackage{eso-pic}																%插入浮水印
\usepackage{graphicx}																%插入圖片
\usepackage[bf,indentfirst,pagestyles]{titlesec}									%定義章節名稱與文字型態
\usepackage {titletoc}																%定義章節名稱
\usepackage{cite}																	%bibliography package
\usepackage{tabularx}
\usepackage{longtable}																%處理長表格
\usepackage{enumitem}
\usepackage{array}
\usepackage{float}
\usepackage{fancyvrb}
\usepackage{fancyhdr}
\usepackage[above]{placeins}
\usepackage{pstricks,pst-node,pst-tree}
\usepackage{wrapfig}


%定義環境變數與巨集指令
%-------------------基本設定------------------------------------%
% 沿用 latex 的一些標點的轉換,如 en-dash 以兩個減號表示
\defaultfontfeatures{Mapping=tex-text}
\XeTeXlinebreaklocale "zh"
\XeTeXlinebreakskip = 0pt plus 1pt
%-------------------------------------------------------------%

%-------------------重新定義章節名稱格式--------------------------%

\titleformat{\chapter}[hang]{\centering\huge\bfseries}{第\CJKnumber{\thechapter}章}{1em}{}
%\renewcommand\chaptername{Chapter\hspace{.5em}\thechapter}
\renewcommand\contentsname{目錄}
\renewcommand\listfigurename{圖目錄}
\renewcommand\listtablename{表目錄}

\titlecontents{chapter}[0pt]{}
    {第\CJKnumber{\thecontentslabel}章\quad}{}
    {\hspace{.5em}\titlerule*[10pt]{$\cdot$}\contentspage}
\titlecontents{section}[2em]{}
    {\thecontentslabel\quad}{}
    {\hspace{.5em}\titlerule*[10pt]{$\cdot$}\contentspage}
\titlecontents{subsection}[4em]{}
    {\thecontentslabel\quad}{}
    {\hspace{.5em}\titlerule*[10pt]{$\cdot$}\contentspage}

% \renewcommand{\refname}{參考文獻}
\renewcommand{\bibname}{參考文獻}			%修改參考文獻的標題名
\titlespacing{\chapter}{0pt}{*0}{*4}		%設定標題與四周的距離
\titlelabel{\thetitle\quad}				%設定章節標題的樣式
\renewcommand{\figurename}{圖}
\renewcommand{\tablename}{表}
%-----------------------------------------------------------------%

%-------------------設定行距--------------------------------------%
\renewcommand{\baselinestretch}{1.6}
%\linespread{1.6}
%設定enumerate等的item間距

%-----------------------------------------------------------------%

%-------------------定義浮水印------------------------------------%
\newcommand\WatermarkPicture{%
   \put(0,0){%
   \parbox[b][\paperheight]{\paperwidth}{%
     \vfill
     \centering
     \includegraphics[width=5cm,keepaspectratio]{watermark_ntut.png}%
     \vfill
     }
   }
}

%-----------------------------------------------------------------%

%------------------圖片巨集---------------------------------------%

%巨集格式
%mygraphic{圖片KeyWord}{圖片註解}{圖片路徑}
\def\myGraphic#1#2#3
{
	\begin{figure}[!htbp]
		\begin{center}
			\includegraphics[width=\textwidth]{#3}
			\caption{#2}\label{#1}
		\end{center}
		
	\end{figure}
}
%-----------------------------------------------------------------%

%------------------小小圖片巨集---------------------------------------%

%巨集格式
%mygraphic{圖片KeyWord}{圖片註解}{圖片路徑}
\def\myGraphicSS#1#2#3
{
	\begin{figure}[!htbp]
		\begin{center}
			\includegraphics[width=2cm]{#3}
			\caption{#2}\label{#1}
		\end{center}
		
	\end{figure}
}
%-----------------------------------------------------------------%

%------------------小圖片巨集---------------------------------------%

%巨集格式
%mygraphic{圖片KeyWord}{圖片註解}{圖片路徑}
\def\myGraphicS#1#2#3
{
	\begin{figure}[!htbp]
		\begin{center}
			\includegraphics[width=6cm]{#3}
			\caption{#2}\label{#1}
		\end{center}
		
	\end{figure}
}
%-----------------------------------------------------------------%

%------------------中圖片巨集---------------------------------------%

%巨集格式
%mygraphic{圖片KeyWord}{圖片註解}{圖片路徑}
\def\myGraphicM#1#2#3
{
	\begin{figure}[!htbp]
		\begin{center}
			\includegraphics[width=8cm]{#3}
			\caption{#2}\label{#1}
		\end{center}
		
	\end{figure}
}
%-----------------------------------------------------------------%

%------------------大圖片巨集---------------------------------------%

%巨集格式
%mygraphic{圖片KeyWord}{圖片註解}{圖片路徑}
\def\myGraphicB#1#2#3
{
	\begin{figure}[!htbp]
		\begin{center}
			\includegraphics[width=12cm]{#3}
			\caption{#2}\label{#1}
		\end{center}
		
	\end{figure}
}
%-----------------------------------------------------------------%

%------------------大大圖片巨集---------------------------------------%

%巨集格式
%mygraphic{圖片KeyWord}{圖片註解}{圖片路徑}
\def\myGraphicBB#1#2#3
{
	\begin{figure}[!htbp]
		\begin{center}
			\includegraphics[width=16cm]{#3}
			\caption{#2}\label{#1}
		\end{center}
		
	\end{figure}
}
%-----------------------------------------------------------------%

%------------------表格巨集---------------------------------------%
\renewcommand{\arraystretch}{1}

%巨集格式
%myTable{Table KeyWord}{Table註解}
\def\myTable#1#2
{
	\begin{table}[!htbp]
	\setlength{\abovecaptionskip}{0pt}
	\setlength{\belowcaptionskip}{10pt}
	\begin{center}
	\caption{#2}\label{#1}
}


\def\endmyTable
{
	\end{center}
	\end{table}
}

%-----------------------------------------------------------------%

%------------------修改圖與表的註解編號格式-----------------------%

\makeatletter
\long\def\@makecaption#1#2{%
  \vskip\abovecaptionskip
  \sbox\@tempboxa{{\bfseries #1}\quad #2}%
  \ifdim \wd\@tempboxa >\hsize
    {\bfseries #1}\quad #2\par
  \else
    \global \@minipagefalse
    \hb@xt@\hsize{\hfil\box\@tempboxa\hfil}%
  \fi
  \vskip\belowcaptionskip}
\makeatother


%-----------------------------------------------------------------%
%------------------修改Description縮排格式--------------------------%
%\makeatletter
%\renewenvironment{description}
%  {\list{}{\leftmargin\z@ \labelwidth\z@ \itemindent-\leftmargin
%   \let\makelabel\descriptionlabel}}
%  {\endlist}
%\makeatother
%-----------------------------------------------------------------%
%------------------Use Case巨集---------------------------------------%
%巨集格式
%useCase{UseCase名稱}{UseCase圖片路徑}
\def\useCase#1#2#3
{
	\subsection{#1}
	\begin{figure}[!htb]
		\begin{center}
			\includegraphics[width=\textwidth]{#2}
		\end{center}
	\end{figure}
}
% \graphicspath{{./picture/eps/}{./picture/png/}}
\graphicspath{{.}}					%告訴Latex去這兩個目錄下找圖檔
%-----------------------------------------------------------------%
%-------------------為了方便所設定的巨集--------------------------%

%-----------------------------------------------------------------%


\begin{document}


 \AddToShipoutPicture{\WatermarkPicture} % 加入浮水印

\pagenumbering{roman} 											%論文目錄部份之頁數為羅馬數字
\chapter*{摘~~要}
\addcontentsline{toc}{chapter}{中文摘要}

%基本資訊

\noindent
論文名稱:跨平台軟體持續整合樣式之已知案例探討			\hspace*{\fill}頁數:47		            \\
校所別:國立台北科技大學~資訊工程系碩士班					  		  						\\
畢業時間:九十九學年度第二學期					\hspace*{\fill}學位:碩士		        \\
研究生:王熙鈞								\hspace*{\fill}指導教授:鄭有進、謝金雲教授\\
\hspace*{\fill}\\
\noindent
關鍵字:跨平台軟體開發,樣式,樣式語言,持續整合\\
\hspace*{\fill}\\
%
\indent

軟體開發團隊實踐持續整合,藉此順利完成軟體開發,這是軟體開發業界逐漸採用的作法。一如軟體開發中眾多問題的解決方法可用樣式加以表達,持續整合的作法亦適宜以樣式表達之。關於持續整合,目前已有若干持續整合樣式及樣式語言被整理提出,用以協助軟體開發團隊在開發跨平台軟體時實踐持續整合。雖然這些跨平台軟體持續整合樣式語言是以特定應用領域的跨平台軟體專案實踐持續整合的經驗為基礎,但要真正成為一個樣式語言,其中之樣式需要有更多的實例。因此本論文以開放原始碼的跨平台軟體專案\textendash\hspace{4pt}Chromium與SWT,作為案例探討的對象,在其中尋找已知案例。我們在實例中已找到了九個樣式的已知案例,足以說明跨平台軟體持續整合樣式語言的適用性。此外,本論文亦藉由於實例中所得到實踐持續整合的經驗與知識,輔以他人已經發表關於持續整合的樣式,整理適當的樣式融入跨平台軟體持續整合樣式語言中,進一步演化此一樣式語言。

											%中文摘要
\chapter*{ABSTRACT}
\addcontentsline{toc}{chapter}{ABSTRACT}

%基本資訊

\noindent
Title:On the known use of a collection of Patterns for Continuous Integration in Cross-Platform Software Development\hspace*{\fill}Pages:47		   \\
School:National Taipei University of Technology						  		   \\
Department:Computer Science and Information Engineering						   \\
Time:June,2011									  \hspace*{\fill}Degree:Master	   \\
Researcher:Si-Jyun Wang							  \hspace*{\fill}Advisor:C.-Y. Hsieh and Y.C.Cheng\\
\hspace*{\fill}\\
Keywords:Cross-Platform Software Development,Pattern,Pattern Language,Continuous Integration\\
\hspace*{\fill}\\
%
\indent
%In this thesis. Taking open source cross-platform software projects as study cases, to discover known uses of the Continuous Integration Patterns for Cross-Platform Software Development. Learning the experiment and knownledge of practicing Continuous Integration from the developers of the study cases, and the patterns about Continuous Integration that already published. And then extracting the essentials from both to help evoluting the Continuous Integration Patterns for Cross-Platform Software Development. Finally, to make the petterns approching completed.
There are well-known success stories that software development teams practicing continuous integration produce quality software. Recently, a pattern language for continuous integration in developing cross-platform software was proposed, helping software development teams to practice continuous integration when developing cross-platform software. The discussion of the known uses of the pattern language is still lacking, however. In this thesis, taking Chromium and SWT\hspace{3pt}\textendash\hspace{3pt}open source  cross-platform projects as study cases, known uses of these patterns are found. In all, we found nine known uses of the patterns, thus proving that the pattern language is a working and living one. In addition, we also consider other known continuous integration patterns and incorporate them to further evolve the pattern language for continuous integration in developing cross-platform software.
											%英文摘要
\chapter*{誌~謝~}
\addcontentsline{toc}{chapter}{誌謝}

三年過了,歷經比一般碩士生更多的修課考驗,終於達到可以畢業的資格,並且完成了該本論文。

在這段期間,曾遭遇令人不愉快的事件,感謝這些人、事、物給我的啟示,讓我可以檢討自己,並且改善自己的個性。

而光明與黑暗並存於這個世界,同時也體驗過與有志一同的人,一起著手一項作業、專案,當順利完成時,心中的激動與成就感,真是令人上癮。

最後,感謝我的家庭、師長、情感上的伴侶、同學、朋友、學弟妹與我已經沒有明顯印象的其他人,一起在這段時光中度過。我得到的經驗與知識,謹記在心。												%誌謝
\newpage


\addcontentsline{toc}{chapter}{目錄}
\tableofcontents 												%引入目錄
\newpage

\addcontentsline{toc}{chapter}{表目錄}
\listoftables													%引入表目錄
\newpage

\addcontentsline{toc}{chapter}{圖目錄}
\listoffigures													%引入圖目錄
\newpage														%換新頁

\pagenumbering{arabic}											%論文內文之頁數為阿拉伯數字

\chapter{緒論}

開發跨平台軟體是一項於技術面、商業策略上都極具挑戰的工作。軟體公司Netscape希望各個平台上的使用者皆可利用其瀏覽器軟體瀏覽網路,所以該公司訴求其瀏覽器可以在不同平台上運行,基於上述原因,於1990年代Netscape佔據80\%瀏覽器市佔率\cite{netscape1990marketshare}。為了達成軟體於不同平台上運行的目標,Netscape定義其瀏覽器軟體為跨平台軟體,並基於同一份程式碼,同時為不同平台使用者開發瀏覽器產品。由Netscape的經驗可知開發跨平台軟體的困難包含:撰寫跨平台程式碼必須留意各個平台上的特性;應該減少專屬於特定平台的原始碼;與於不同平台上執行測試將更加耗時\cite{netscapecrossplatform}。雖然Netscape最終因為軟體生命週期管理不當、微軟競逐瀏覽器市場等因素,而逐漸式微,但是該公司對於跨平台軟體開發的經驗依然值得借鏡。

在1980年代,基於不同的商業策略,雖然由微軟所開發的Word可以同時支援Windows, Mactintosh平台,但是微軟並不是以開發跨平台軟體的手段達成,而是利用p-code\cite{microsoftpcode}的特性迴避開發跨平台軟體\cite{netscapecrossplatform}。在不需做任何修改的條件下,同一份以p-code呈現的可執行檔案可以在上述兩個平台執行,在原始碼極少需要因應不同平台而有不同實作的有利條件下,微軟將原始碼編譯為p-code同時於兩個平台上佈署,代價有二:(1)Word在Mactintosh平台上的執行速度較慢。(2)Word在Macintosh平台上的GUI功能比較匱乏。由上述內容得知,微軟的Word產品並不能算是跨平台軟體,至今微軟Office系統雖然可以於Mac OS X上執行,但是該版本Office系統是由Windows平台的Office移植過去的。綜合微軟於不同平台上經營Word軟體的策略,與Netscape發展跨平台瀏覽器的經驗,足以說明跨平台軟體開發在技術面、商業策略上的困難與挑戰。

本論文僅就開發跨平台軟體的技術面議題論述。Martin Fowler提出持續整合概念,並且主張發展一個產品的過程中,無論是軟硬體,都不能將整合工作延遲至專案開發階段的末期才開始進行,而必須於專案初期就進行整合,並且在交付產品前都不能停止\cite{martinfowlerci}。基於這樣的精神,注重及早交付產品的敏捷式軟體開發方法(Agile Methods)將持續整合視為最佳實踐\cite{xpagilemethods}。在執行持續整合時,首重確實實行各種必須涵蓋於其中的步驟,例如為了維持軟體系統品質,於整合流程中,敏捷式軟體開發團隊涵蓋各種驗證軟體品質的步驟。此外相關經驗,例如執行持續整合流程時各種錯誤排除的經驗如何於團隊成員間擴散,亦為實踐持續整合必須考慮的事項。在市面上有許多持續整合的工具,從開放原始碼到商用軟體,為數眾多,然而只依賴工具並無法傳達如何實踐持續整合。軟體開發團隊若未體悟工具無法根除執行持續整合本質上的問題,且團隊未建立起積極面對整合議題的態度,則持續整合易淪為形式。

跨平台軟體開發團隊執行持續整合流程,用以降低於不同平台上進行測試的成本。於持續整合流程中必須涵蓋各種形式的自動化測試,這樣的作法將對跨平台軟體開發帶來衝擊,因為於持續整合流程中,執行過多的測試將造成開發團隊簽入原始碼的過程冗長,如此間接使得團隊成員執行持續整合的意願降低\cite{teampace}。面對上述的問題,開發團隊需能以有效途徑找到解決方法,並與其他開發團隊分享這些經驗,以幫助其他開發團隊執行持續整合。在這方面,樣式(Pattern)為一個公認有效的途徑。

樣式於特定情形、條件下用以解決於其中發生的問題,因為發展樣式的過程必須嚴謹,所以樣式合適用來傳達執行持續整合時的經驗與知識,而實踐持續整合並無法以單一樣式表達其中經驗,所以必須將其中各個經驗發展成樣式以利傳達,並且連結樣式形成樣式語言,用以描述實踐持續整合之整體經驗與知識。於知名樣式網站Portland Pattern Repository\cite{portlandcipatterns}上,十一個關於持續整合的樣式已經發表,然而其中所刊載之樣式多半非正規形式呈現,僅為各個開發者實踐持續整合時的經驗分享。

C.-Y. Hsieh等提出跨平台軟體持續整合樣式語言\cite{crossplatformcipatterns},並以正規形式呈現其中樣式,期望跨平台軟體開發團隊利用該樣式語言,順利實踐持續整合,藉此完成跨平台軟體之開發。 雖然藉由特定領域軟體開發實踐持續整合的經驗,C.-Y. Hsieh等提出的跨平台軟體持續整合樣式語言,尚需要更多的實例加以實證之。

本論文以軟體建置、檔案目錄結構、架構設計、專案社群文件四個構面為途徑探討開放原始碼專案Chromium與SWT,期望於其中得到跨平台軟體持續整合樣式之已知案例,用以確定樣式語言存在之真實性與其有效性。 此外藉由跨平台軟體開發團隊對於實踐持續整合的經驗與知識,輔以他人已經發表過持續整合樣式,將合適的樣式與經驗融入跨平台軟體持續整合樣式語言中,進一步演化此一樣式語言。

本論文分成五個章節。第一章為緒論,介紹研究背景與動機、研究目標、論文組織與架構。第二章為背景介紹,介紹關於跨平台軟體開發、持續整合、樣式語言、跨平台軟體持續整合樣式語言等知識。第三章為實例介紹,針對真實世界中的跨平台軟體專案,就跨平台軟體持續整合樣式語言的分類,分別舉出實際案例說明。第四章為演化跨平台軟體持續整合樣式語言,針對真實案例和他人已經發表的樣式,論述其融入跨平台軟體樣式語言的根據。第五章為結論,介紹本論文的貢獻與未來可能的研究方向。



\chapter{背景介紹}
本章各小節分別介紹跨平台軟體開發、持續整合、樣式語言與跨平台軟體持續整合樣式語言與Portland Pattern Repository整理的持續整合樣式\cite{portlandcipatterns}作介紹。

\section{跨平台軟體開發}

我們以瀏覽器軟體為例,說明開發跨平台軟體的目的。為了要滿足不同平台上瀏覽網路的使用者,各家瀏覽器供應商除了微軟以外\footnote{因為搭載視窗作業系統的上網裝置還是達到80\%以上的市佔率。\cite{osmarketshare}},都將其自家產品定義為跨平台軟體,期望可以提高在瀏覽器市場的市佔率。瀏覽器軟體的應用領域\cite{applicationdomain}是網際網路應用與HTML等規格,這些規格無關於平台特性,但是瀏覽器供應商必須同時於多個平台上開發同一個產品。

論述開發跨平台軟體的挑戰之前,必須定義跨平台軟體開發(Cross-Platform Development)與移植(Porting)的差異點\cite{crossplatformpotingdiff}。移植指的是先在一個A平台上有軟體實作,之後,再另一個B平台上,將這份軟體實作修改,使其可以在B平台上運作,基本上,我們將這個軟體在A平台與B平台上的code base視為兩份不同的程式碼。然而,跨平台軟體的程式碼有些部分是通用、與平台無關的,而有一些是針對平台的特性去撰寫的程式。此外必須同時匯集多組專精特定平台的開發者於同一個code base開發功能,各個平台上的專家,彼此之間所熟悉的開發工具程式、除錯工具程式、程式撰寫習慣不同,甚至是分屬不同時區、地區的開發團隊。最後,同一個釋出版本,會有各個平台的執行檔被釋出。

開發跨平台軟體的挑戰,主要分成設計面、需求面與流程面三方面介紹。

在設計面,應該把與平台相關的程式碼分離開,並提供一個抽象化介面來封裝與平台相關的實作,其他與平台無關的程式碼相依於此介面,而不是個別平台的實作\cite{gofdesignpattern}\cite{slogan}。

在需求面上,開發者必須面對在A平台上可以輕易實作出來的功能,在B平台上卻無法以相同方法實現之困境,例如開發者於iOS平台上可輕易地利用iOS提供的API以HTTPS存取網站,但於Android平台上尚無法利用Android提供的API以HTTPS存取網站。此外為了避免依賴只能於特定平台上運行的API X,開發者必須使用由團隊設計於各個平台通用的介面,但是上述只限定於某平台的API X卻能提供最佳效能。於該平台上,此限制導致跨平台軟體的執行效能較其他可以運用API X的軟體稍差\cite{netscapecrossplatform},例如在Windows平台上,開發者可以呼叫GPU供應商所提供專屬於Windows平台的API,用以利用GPU所提供的運算能力強化影片播放之流暢度,但是這樣的功能利用於各個平台通用的介面無法達成,且於其他平台上,開發者無法運用該專屬某平台的API。

在流程面,開發人員於不同平台上撰寫程式碼,並將程式碼簽入同一程式碼版本控制系統中,開發者A因為撰寫適用於特定平台的程式碼,而造成軟體無法在其他平台建置,該開發者在不知情的狀況下簽入版本控制系統,最後導致其他開發者無法進行開發工作。此外,由於必須時常對多平台進行測試,需要耗費倍數於對單一平台測試的資源進行測試\cite{netscapecrossplatform}。

\section{持續整合}

對於執行整合工作,軟體工程師往往無法準確估計時程,這是因為個別模組可以獨立地正常運作,但是和其他模組整合在一起時,整個軟體系統卻無法運作。此特性使得整合工作具備高風險。一般作法中軟體工程師在接近軟體釋出的階段,才開始進行整合工作。這意味著一旦有整合失敗的情形發生,延期釋出軟體是難以避免的。風險越高的工作需要越早、越頻繁被執行,所以Steve McConnell\cite{rapiddevelopment}提倡每日建置與冒煙測試的實踐概念,於此概念中整合工作至少每天被執行一次。Martin Fowler提出持續整合的觀念\cite{martinfowlerci},持續整合概念主張以自動化的方式對於軟體進行編譯、測試、靜態分析、產生可交付產品的流程、產出對於程式碼進行測試後的報表,並且在一天當中進行不只一次的整合工作,用以分散、降低執行整合工作的風險,進而有效減輕軟體開發風險。

Paul M. Duvall等對於軟體專案的風險進行分類,主要有下列4個原因:缺乏可交付的軟體;卻乏找尋軟體缺陷的機制;缺乏對於軟體專案狀態進行了解的機制;與低落的軟體品質\cite{continuousintegration2007}。本論文綜合上述原因論述利用實踐持續整合可以減輕上述風險,進而降低開發軟體的成本的原因。
\begin{enumerate}
\item 缺乏可交付的軟體:於持續整合流程中,開發團隊會定義如何產生可以交付給使用者使用、提供測試人員進行測試的安裝檔案,並且於每一次建置整合時產出該檔案。
\item 缺乏找尋軟體缺陷的機制:於持續整合流程中,開發團隊會針對軟體進行一連串自動化測試用以尋找軟體缺陷。
\item 缺乏了解軟體專案狀態的機制:於持續整合流程中,開發團隊會將建置整合的結果利用報表呈現,該報表即軟體專案狀態之具體呈現。
\item 低落的軟體品質:於持續整合流程中,開發團隊進行各種自動化測試與靜態分析。有助於軟體品質的維護與提昇。
\end{enumerate}

當執行整合工作的頻率增加,即新的程式碼簽入版本控制系統中即執行一次整合工作,而且在軟體開發週期中沒有停止並永續地被執行,如此的頻繁程度則可以稱為持續,若軟體開發團隊落實上述的實踐,則宣稱該團隊實踐持續整合。此外當編譯錯誤、測試不通過、測試用的資料庫有問題,總之任何使建置整合工作出現錯誤的情況,即可稱為發生Broken Build,當上述情形發生時,軟體開發團隊必須儘早排除狀況,以維護軟體的整體品質。

持續整合可以紓緩開發跨平台軟體時,對於流程面的衝擊。每一次程式碼簽入版本控制系統,必須進行一次整合,並且在所有預定要佈署的平台上都進行整合,使開發團隊能儘早解決整合過程發生的問題。此外於各個平台上,開發跨平台軟體團隊進行測試所衍生的總成本以數量A表示,開發非跨平台軟體的測試成本以數量B表示,數量A是數量B的許多倍,上述情況是開發跨平台軟體時的主要問題,所以整合過程必須包含自動化的測試,藉助自動化測試的幫助得以有效降低成本。

軟體開發團隊執行持續整合除了依賴工具,用以執行自動化的建置與驗證流程外,尚需要落實各種最佳實務。Martin Fowler於\cite{continuousintegration2007} \textendash\hspace{3pt}導讀中提到執行持續整合會對軟體開發流程造成副作用,而軟體開發團隊會因為副作用而放棄實踐持續整合。例如於整合流程中進行耗費時間的測試工作,軟體開發團隊於簽入程式碼前,於自身的工作環境中,必須執行一次整合,當整合於本地端順利完成,開發者才可以簽入程式碼進入版本控制系統。如果該次整合中包含冗長的測試,開發團隊因為無法忍耐長時間的測試過程,因而放棄落實整合工作。除此之外,Fowler提示軟體開發團隊落實持續整合除了借重工具輔助外,尚需要實踐各種最佳實務,才能使得軟體開發團隊繼續實踐持續整合,以達成持續整合的永續執行特性。


\section{樣式與樣式語言}

樣式語言可以描述落實最佳實務之經驗與知識。對於樣式的定義,節錄A Pattern Language(1977)\cite{apatternlanguage}一書關於樣式的定義\footnote{請見該本書序言的第10頁。},來做說明。

\textit{“The elements of the language are entities called patterns. Each pattern describes a problem which occurs over and over again in our environment, and then describes the core of the solution to the problem, in such a way that you can use this solution a million times over, without ever doing it the same way twice.”}

所以樣式是針對特定情境下,所發生問題的解法,而且是可以被重複使用的解法。通常要呈現一個樣式可以用圖~\ref{patternformat}形式表達。
\begin{figure}
\fbox{
\parbox{14.5cm}{
\begin{description}
\setlength{\parsep}{0ex minus5ex}
\setlength{\itemsep}{0ex minus5ex}
\setlength{\topsep}{0ex plus5ex}
\item Name:樣式命名。
\item Context:於一個情境中,樣式為了解決問題被發展。
\item Forces:於情境中所出現的現象,互相衝突的現象形成問題。
\item Problem:於情境中出現的問題。
\item Solution:在特定情境中解決問題的方法。
\item Resulting Context:解法套用後,情境的描述。
\item Related Patterns:與樣式相關的其他樣式。
\end{description}
}
}
\caption{樣式之呈現形式}
\label{patternformat}
\end{figure}
對於樣式與樣式語言之間的關係,舉於The Timeless Way of Building(1979)\cite{thetimelesswayofbuilding}出現的例子\footnote{前見該本書的第313頁。},解釋兩個樣式之間有連接關係存在時,此關係的解讀,請見圖~\ref{pattern-connection}。
\myGraphicSS{pattern-connection}{樣式A 與 樣式B之間關係解釋圖示}{pattern-connection.eps}
樣式A 如果要解決預定要解的問題,必須依賴樣式B,則樣式B 解決了一部分樣式A要解的問題;如果樣式B未能解決這一部分的問題,樣式A就無法解決預定要解決的問題。樣式之間必須存在結構化網絡關係,樣式才可能形成樣式語言。樣式語言可以視為有一個目標、要解決一個規模更大的問題、要平衡為數更多且更加複雜的現象(Forces),處在樣式語言網絡中的樣式,一同解決樣式語言要解決的問題,最後最後使得失衡的現象(Forces)重新達成平衡。
 
\section{跨平台軟體持續整合樣式語言}

基於傳達開發跨平台軟體時實踐持續整合的知識與經驗,C.-Y. Hsieh等提出跨平台軟體持續整合樣式語言;其目標主要為降低跨平台軟體開發成本、克服實踐持續整合的副作用,永續實踐持續整合。藉上述兩點幫助軟體開發團隊開發跨平台軟體。跨平台軟體持續整合樣式語言,將樣式分成三大類,分別是Project, Build, Good Habit。可以將此三大類之樣式建立網絡關係,形成樣式語言\footnote{Good Habit類只有一個樣式,基本上無法稱為一個樣式語言,後續工作將演化Good Habit 範疇樣式語言形成一個形式上與功能上完整的樣式語言。}。三大類樣式語言要解決的問題分別為。
\begin{enumerate}
\item 軟體開發團隊於開發跨平台軟體時,在實踐持續整合的前提下,如何依照separation of concern和module decomposition的原理\cite{satextbook},將專案分解成數個子專案。
\item 跨平台軟體開發團隊於實踐持續整合時,在避免Broken Build的前提下,如何定義建置軟體的工作流程。
\item 跨平台軟體開發團隊於實踐持續整合時,如何因應Broken Build的發生。
\end{enumerate}
\myGraphicM{project-category}{Project 範疇樣式語言網絡圖示}{project-catgory-pattern-language-network.eps}

圖~\ref{project-category}描述Project 範疇樣式語言網絡關係。僅針對Interface Project, Cross-Platform Project, Native Project, Single Shared Library Project四個樣式之間的關係作說明。
\begin{description}
\item Cross-Platform Project與Interface Project的關係:與平台無關的程式碼,必須透過處於Interface Project中的介面定義,間接去使用與平台相關的實作。Cross-Platform Project樣式的功能要趨近完整,必須相依於Interface Project樣式本身的能力。
\item Native Project與Interface Project的關係:與平台有關的實作程式碼,如果要被平台無關的程式碼所使用,必須依循著介面的定義實作,與平台無關的程式碼才能透過介面使用到這些實作。Native Project樣式的功能要趨近完整,必須相依於Interface Project樣式本身的功能。
\item Interface Project, Cross-Platform Project, Native Project與Single Shared Library Project的關係:歸屬於上述三種專案的程式碼參考第三方的軟體函式庫時,因為Single Shared Library Project解決了第三方軟體函式庫管理的問題,所以程式碼參考第三方軟體函式庫的問題,必須依賴Single Shared Library樣式提供的能力解決。此外Single Shared Library Project樣式尚且描述以下實踐。因為開發者需要控管上述三種子專案之產出物版本,所以必須替產出物建立版本控制機制,開發者將產出物簽入Single Shared Library專案之版本控制庫。於開發者所處之開發環境,在建置相依於子專案的專案時,開發者僅更新Single Shared Library Project即可建置專案,專案亦僅需相依於Single Shared Library Project,不需要從頭建置專案所需之子專案產物,用以加速建置流程。但在實踐上述機制的前提下,每次執行建置時,我們無法保證專案相依最新版本之子專案產出物,所以在釋出軟體產品時,於建置流程上必須定義重新建置所有子專案之機制,用以確保所有子專案產出物以最新版本佈署於客戶端。
\end{description}
\myGraphicS{build-category}{Build 範疇樣式語言網絡圖示}{build-catgory-pattern-language-network.eps}

在跨平台軟體持續整合樣式語言提出時,尚未表達Build 範疇樣式語言網絡關係,本論文提出該樣式語言的網絡關係,請見圖~\ref{build-category}。Remote Private Build樣式與Cross-Platform Integration樣式、Cross-Platform Integration樣式與Integration Workflow樣式、Local Private Build樣式與Integration Workflow樣式,上述樣式之間分別存在關係。

將上述關係作為範例說明,其中情境為原始碼簽入版本控制系統前。\begin{description}
\item Remote Private Build與Cross-Platform Integration樣式之間關係:Remote Private Build樣式規範開發者將開發端的原始碼變更置放於持續整合系統進行建置整合流程,此流程必須在所有產品預設要佈署的平台上執行,Cross-Platform Integration樣式用以描述於各個平台上執行建置整合的觀念。所以Remote Private Build樣式要發揮完善功能,必須相依Cross-Platform Integration樣式。
\item Cross-Platform Integration與Integration Workflow樣式之間關係:Cross-Integration樣式描述於各個平台上針對同一個程式碼變更進行建置整合,Integration Workflow樣式用以描述上述於各個平台上進行之建置整合流程。所以Cross-Platform Integration樣式要發揮完善功能,必須相依Integration Workflow樣式。
\item Local Private Build樣式描述於各平台開發者之工作環境中進行建置整合流程,於各個平台上進行之建置整合流程可利用Integration Workflow樣式加以描述。所以Local Private Build要發揮完善功能,必須相依Integration Workflow樣式。
\end{description}

\section{Portland Pattern Repository 持續整合樣式語言}
知名的樣式語言網站Portland Pattern Repository\cite{portlandcipatterns},記載了11個關於持續整合的樣式,Single Unified Build Script, Use One Code Line, Commit Early And Often, Reduce Size Of Check In, Continuous Integration Relentless Testing, Single Integration Point, Single Release Point, Collective Code Ownership, Dont Integrate Mid Task, Update Often Commit Only After Testing, 與Incremental Integration,這些樣式以風格自由且沒有固定形式的方式表達。此外有一些樣式與Software Configuration Management\cite{scm}的最佳實踐有關,如 Use One Code Line、Commit Early And Often、Reduce Size Of Check In等。有一些樣式與Extreme Programming\cite{xp}提倡的最佳實務有關,如Collective Code Ownership。經由上述,我們推斷實踐持續整合與Software Configuration Mangement、軟體開發流程有明顯關聯。本論文將從這些前人經驗累積所得的樣式,擷取合適融入跨平台軟體持續整合樣式語言的樣式,並且演化該樣式語言。

\chapter{跨平台軟體持續整合樣式語言的實例}
\vfill
\textit{In order to discover patterns which are alive we must always start with observation.} \begin{flushright}Christopher Alexander, The Timeless Way of Building(1979), p.254.\end{flushright}
\vfill

本章節將以Chromium\cite{chromiumproject}與SWT\cite{swtproject}專案為實例,探討跨平台軟體持續整合樣式的已知案例。首先本論文提出探討跨平台軟體持續整合樣式已知案例之研究方法,接著分別探討屬於Project與Build範疇樣式之已知案例,關於Project與Build範疇樣式的介紹請參閱2.4節。接者描述Good Habit範疇樣式中,現階段唯一的樣式Single Responsible Person是如何被實踐的,並討論產生監控持續整合狀態負責人員的機制。最後討論以研究方法進行案例探討之結果。

\section{研究方法}

本論文以軟體建置、檔案目錄結構、架構設計、專案社群文件等途徑觀察探討樣式的已知案例,請見圖~\ref{observation-view}。%界定專案是此方法主要探討的核心,以不同的構面觀察將得到不同認知。
\begin{description}
\item 軟體建置:軟體建置將原始碼轉換成可執行檔,原始碼依照架構設計被歸類至不同的目錄中。隸屬不同子目錄的程式碼經過編譯形成一個元件,而各個元件以一定順序進行連結以形成最終的可執行檔。因為建置腳本\footnote{Apache Ant build script, Microsoft Visual Studio Solution files, Xcode Project files等皆可歸類為建置腳本。}記錄軟體建置步驟、概念上相近的程式碼形成元件的關係,所以提供我們識別Installation Project, Patch Project, Native Project, Cross-Platform Project, Interface Project的線索。
\item 檔案目錄結構:依照separation of concern與module decomposition原則,軟體開發團隊將意義上相似、可以聚合形成模組的程式碼分類並置放於不同目錄,所以介面與實作應該置放在不同的目錄。在意義上相似的程式碼所聚集成的目錄,經由此構面觀察可視為一個專案。
\item 架構設計:基於檔案目錄結構構面並考慮架構設計議題,本構面針對處於不同目錄之程式碼所產生的關連加強論述。在設計系統架構時,利用介面分離與平台有關的實作、與平台無關的通用程式碼,分離後上述三種類型的原始碼應該分別置於不同目錄,但並不能強制規範進行分離的作法。但在僅需新增對應平台的實作而不更動介面的前提下,為了方便新增對另一種平台的支援,介面、與平台相關的程式碼必須被強制分離至不同子專案中。
\item 專案社群文件:因為成員分散於世界各地,所以開放原始碼專案社群文件必須記錄有關軟體建置、檔案目錄結構、架構設計等等知識,方便新進成員儘快了解該專案。對於各類討論持續整合相關議題之書籍與網站,以專案社群文件構面進行觀察時,本論文亦將上述內容納入該構面中。
\end{description}

上述四種途徑有互相依賴的現象,實際進行觀察時,觀察者不可能只利用一種途徑探討樣式之已知案例。通常首先從建置軟體的構面進行觀察,因為以持續整合的觀點出發,最直接被關心的是軟體如何被編譯建置。再來我們關注概念上相似的程式碼如何被分類、置放於目錄結構中,於其中可以定位與平台有關、無關之專案。接著我們關注分屬不同類型專案之程式碼間的關連,找尋是否有扮演介面角色之程式碼,並確定該類程式碼、與平台有關之程式碼、與平台無關之程式碼之間的關係,以上描述有賴架構設計構面之觀察才可實現。最後,輔以專案社群文件購面進行觀察,可以補足或加速以上述三種構面進行觀察不足之處。

\myGraphicM{observation-view}{用以觀察已知案例的四個構面}{observation_view.eps}

\section{探討Project類樣式語言以\textendash\hspace{4pt}Chromium與SWT專案為實例}
Chromium專案是一個跨平台開放原始碼專案,以C, C++, Objective-C程式語言實作一個Web瀏覽器,主要支援Linux, Mac OS X, Windows等平台。在Mac OS X平台上,與UI相關的程式碼以Objective-C程式語言實作,與其他平台共用之程式碼則使用C, C++程式語言實作。這是由Google所贊助的專案,Google正式釋出的瀏覽器是以Chrome命名,一般來說Chromium會提供比較新穎的功能,這些功能大概一段時間後,Google才會在釋出新版本的Chrome瀏覽器時,將這些功能加以涵蓋。Google 藉著Chromium專案在瀏覽器市場中獲得10\%左右的市佔率\cite{browsermarketshare}。

SWT專案發展出一套用以開發Java跨平台程式的GUI框架,並以Java, C程式語言實作該框架。在不同平台上,Java程式設計師利用SWT GUI框架撰寫的GUI程式,無論就外觀或是效能方面,都能十分接近以低階語言\footnote{例如C語言。}呼叫原生API\footnote{例如於Windows平台上的Win32 API。}撰寫出來的GUI程式。對以Java語言撰寫GUI程式的程式設計師而言,基於SWT GUI框架開發跨平台之Java GUI程式,不需要區分程式要在哪一個平台上執行,只需要知道程式是在JVM\cite{jvm}上執行,僅與Java執行環境相依即可。要達成這個目的必須依賴JNI技術\cite{jni},先利用Java程式語言定義一些介面,與平台無關的程式碼利用這些介面呼叫與平台有關的系統API;此外呼叫系統API的功能必須以C程式語言實作。

\subsection{探討Chromium專案中Project 範疇樣式語言的已知案例}

首先從建置軟體的構面出發進行觀察,於Mac OS X系統上,Chro\-mium\-專案的建置腳本即Xcode IDE專案檔,利用Xcode IDE\cite{xcode}開啟代表實例的主要專案檔案chrome.\-xcode\-project,實際開啟後,發現實例中所有相關的程式碼在Xcode中是以名為chrome的專案涵蓋,其下有許多子目錄與子專案,如表~\ref{projectlistview}。各個專案中定義Targets,於其中定義由哪些檔案經過編譯後形成動態連結檔案、執行檔;以下將舉chrome與content兩個子專案為例討論其中的Target,請見圖~\ref{deploymentview-content}與圖~\ref{deploymentview-chrome}。

首先討論content子專案,置放於content\-/\textit{Source}\-/\textit{browser}\-, content\-/\textit{Source}\-/\textit{common}\-子目錄中的程式碼,經過編譯形成兩個動態連結檔案libcontent\_browser.a, libcontent\_common.a。以形成一個動態連結檔案的數個原始碼視為一完整概念,則content\-/\textit{Source}\-/\textit{browser}\-, content\-/\textit{Source}\-/\textit{common}\-子目錄視為一專案。

再來討論chrome子專案,置放於render\_host, tab\_contents子目錄中的程式碼與其他置放在chrome\-/\textit{Source}\-/\textit{browser}\-目錄底下的程式碼一同編譯形成一可執行檔案,以形成一個可執行檔案的數個原始碼視為一完整概念,則chrome\-/\textit{Source}\-/\textit{browser}\-子目錄視為一專案。

以軟體建置的構面觀察尚無法明確的指出建置順序與子專案間的關係,很難清楚界定與平台有關、無關與提供介面的子專案,所以必須以設計面的觀點輔助探討。此外以此構面觀察所界定的專案,與經由檔案目錄結構與架構設計構面觀察所界定的專案,我們可以預期兩者不一定完全一致。

\myGraphicB{deploymentview-content}{以建置軟體的構面觀察content子專案}{deploymentview.eps}

\myGraphicB{deploymentview-chrome}{以建置軟體的構面觀察chrome子專案}{deploymentview-chrome.eps}

\myGraphicB{chromium-mvc}{以共同介面建立與平台有關的View}{chromium-mvc.eps}


\begin{myTable}{subprojectfolderview}{Chromium專案的子專案與子目錄}
\begin{tabular}[width=\textwidth]{|c|c|c|c|}
\hline
\multicolumn{4}{|c|}{專案名稱}\\
\hline
\multicolumn{4}{|c|}{chrome}\\
\hline
\hline
\multicolumn{4}{|c|}{\textit{子目錄}}\\
\hline
\textit{Source}&\textit{Intermediates}&\textit{Product}&\textit{Frameworks}\\
\hline
\textit{Projects}&\textit{Build}&&\\
\hline
\hline
\multicolumn{4}{|c|}{\textit{Projects}所涵蓋的子專案}\\
\hline
JavaScriptCore&debug\_stub&gfx&libjingle\\
\hline
WebCore&default\_plugin&gmock&libjpeg\\
\hline
Webkit&desc&goobsdiff&libpng\\
\hline
app&dynamin\_annotations&gpu&libsrtp\\
\hline
base&expat&gtest&libwebp\\
\hline
build\_angle&experimental&hunspell&libxml\\
\hline
bzip2&ffmpeg&iccjpeg&libxslt\\
\hline
cacheinvalidation&flac&icu&media\\
\hline
chromotocol&flash\_player&ime&mesa\\
\hline
cld&gio\_wrapper&ipc&modp\_b64\\
\hline
cloud\_policy\_codegen&gio&jingle&nacl\_base\\
\hline
content&gdb\_rsp&libevent&net\\
\hline
service\_runtime\_x86\_32&nss&remoting&ocmock\\
\hline
service\_runtime\_x86&plugin&npapi&skia\\
\hline
service\_runtime&printing&sdch&speex\\
\hline
safe\_browsing&platform&ppapi&sqlite\\
\hline
platform\_quality&ots&srpc&ssl\\
\hline
pdfsqueeze&protobuf&sync\_proto&trace\\
\hline
ui&ui\_strings&v8&validator\_x86\\
\hline
webkit&xz&zlib&\\
\hline
\end{tabular}
\label{projectlistview}
\end{myTable}


\begin{figure}

	\psscalebox{0.8}{\psset{%linecolor= bisque,%
			nodesep=2pt,%,
			 treemode=D,%
			 levelsep=*1.5cm}
\pstree{\Tr{chrome.xcodeproject}}
			 {\pstree{\Tr{chrome}}{
			    {\pstree{\Tr{\textit{Source}}}
			 	{\pstree{\Tr{\textit{browser}}}
					{\pstree{\Tr{\textit{tab\_contents}}}
					  {\Tr{\underline{render\_view\_host\_delegate\_helper.cc}}
					  }
					  \pstree{\Tr{\textit{render\_host}}}
					  {\Tr{\underline{render\_widget\_host\_view\_*}}
					  }
					  \Tr{...}
					}
				   \Tr{...} 
				}
				
			   }
			 	\pstree{\Tr{content}}
				   {\pstree{\Tr{\textit{Source}}}
					{\pstree{\Tr{\textit{browser}}}
			 			{\pstree{\Tr{\textit{render\_host}}}
					  	{\Tr{\underline{render\_widget\_host\_view.h}}
					  	%{\Tr{render\_widget\_host\_view\_gtk.cc}}
					  	%{\Tr{render\_widget\_host\_view\_mac.mm}}
					  	}
					  	\Tr{...}
					}
				 	{\Tr{...}}
					}
				          \Tr{...}	
				  }
				  
			 \Tr{...}
			 }
			}
			}
		
\begin{center}
\caption{專案結構與目錄結構 - Chromium,其中子目錄以斜體字體標示,原始碼檔案以底線標示,子專案以正常字體標示。}
\label{fig-folderstructure}
\end{center}
\end{figure}

\begin{figure}
\linespread{0.8}
\begin{Verbatim}[numbers=left,framesep=1mm,numbersep=-12pt]
	節錄自render_widget_host_view.h
	// RenderWidgetHostView is an interface 
	// implemented by an object that acts as
	// the "View" portion of a RenderWidgetHost.
	// The RenderWidgetHost and its
	// associated RenderProcessHost own
	// the "Model" in this case which is the
	// child renderer process.
	// The View is responsible for receiving
	// events from the surrounding environment
	// and passing them to the RenderWidgetHost,
	// and for actually displaying the content
	// of the RenderWidgetHost when it changes.
	Class RenderWidgetHostView{
	    public:
	        virtual ~RenderWidgetHostView();
	        
	        // Platform-specific creator.
	        // Use this to construct new 
	        // RenderWidgetHostViews
	        // rather than using 
	        // RenderWidgetHostViewWin 
	        // & friends.
	        ...
	        static RenderWidgetHostView* Create
	            ViewForWidget(RenderWidgetHost
	            View* widget);
	          ...
	  }
\end{Verbatim}
\caption{節錄程式碼 part1: 介面定義}
\label{interfacedef}
\end{figure}
\begin{figure}
\linespread{0.8}
\begin{Verbatim}[numbers=left,framesep=1mm,numbersep=-12pt]
	節錄自render_widget_host_view_win.cc 
	RenderWidgetHostView* RenderWidgetHostView
	::CreateViewForWidget(RenderWidgetHostView*
	widget){
	    return new 
	    RenderWidgetHostViewWin(widget);
	}
\end{Verbatim}
\caption{節錄程式碼 part2: Windows平台之實作}
\label{windowsimple}
\end{figure}
\begin{figure}
\linespread{0.8}
\begin{Verbatim}[numbers=left,framesep=1mm,numbersep=-12pt]
	節錄自render_widget_host_view_linux.cc
	RenderWidgetHostView* RenderWidgetHostView
	::CreateViewForWidget(RenderWidgetHostView*
	widget){
	    return new 
	    RenderWidgetHostViewGtk(widget);
	}
\end{Verbatim}
\caption{節錄程式碼 part3: Linux平台之實作}
\label{linuximpl}
\end{figure}
\begin{figure}
\linespread{0.8}
\begin{Verbatim}[numbers=left,framesep=1mm,numbersep=-12pt]
	節錄自render_widget_host_view_mac.mm
	RenderWidgetHostView* RenderWidgetHostView
	::CreateViewForWidget(RenderWidgetHostView*
	widget){
	    return new 
	    RenderWidgetHostViewMac(widget);
	}
\end{Verbatim}
\caption{節錄程式碼 part4: Mac平台之實作}
\label{macimpl}
\end{figure}

\begin{figure}
\linespread{0.8}
\begin{Verbatim}[numbers=left,framesep=1mm,numbersep=-12pt]
	節錄自render_view_host_delegate_helper.cc
	RenderWidgetHostView* RenderViewHostDelegateViewHelper::
	CreateNewWidget(int route_id, 
	WebKit::WebPopupType popup_type, 
	RenderProcessHost* process) {
		RenderWidgetHost* widget_host =
		new RenderWidgetHost(process, route_id);
		RenderWidgetHostView* widget_view =
		RenderWidgetHostView::
		CreateViewForWidget(widget_host);
		// Popups should not get activated.
		widget_view->set_popup_type(popup_type);
		// Save the created widget associated with
		// the route so we can show it later.
		pending_widget_views_[route_id] = widget_view;
		return widget_view;
	}
\end{Verbatim}
\caption{節錄程式碼 part5: 與平台無關之程式碼呼叫介面}
\label{crossplatformcall}
\end{figure}

研究Chro\-mium專案網站上的設計文件\cite{chromiummvc},發現Chro\-mium專案在設計上使用Model-View-Controller架構\cite{modelviewcontroller},十分清楚地區分與平台有關的實作、與平台無關的程式碼、介面。圖~\ref{chromium-mvc}表達與View相關的程式碼具有與平台相關的特性,像是專屬於視窗平台之實作Render\-Widget\-Host\-View\-Win,該類別實作於Render\-Widget\-Host\-View定義的介面。如有一與平台無關的程式碼檔案必須參考Render\-Widget\-Host\-View\-Win,必須避免在該程式碼檔案中,撰寫一些判斷不同平台行為的程式碼。因此需要提供一個介面,讓與平台無關的程式碼可以依賴該介面,而不需依賴專屬不同平台的實作。根據這個作法可以在類別Render\-Widget\-Host\-View觀察到介面的存在,而這個類別的定義存在於render\_widget\_host\_view.h中。這個檔案是置放於content子專案中的browser子目錄中的render\_host目錄,請見圖~\ref{fig-folderstructure}。
而對應不同平台的實作如Windows, Linux, Mac\-\ OS\-\ X分別存在render\_\-widget\_\-host\_\-view\_\-win.cc, render\_\-widget\_\-host\_\-view\_\-linux.cc, render\_\-widget\_\-host\_\-view\_\-mac.mm檔案中,請見圖~\ref{windowsimple}, ~\ref{linuximpl}, ~\ref{macimpl}。實作檔案置放於chrome專案中的 browser 目錄底下的render\_host目錄,請見圖~\ref{fig-folderstructure}。
介面指的是Create\-View\-For\-Widget\-(Render\-Widget\-Host* widget)這個Method,請見圖~\ref{interfacedef}的第二十五行,此外請注意該Method是定義為static Method,且亦為Factory Method之實踐\cite{gofdesignpattern},用以抽象化專屬於不同平台之RederWidgetHostView實體的創造過程。

接著,我們發現定義於render\_view\_host\_delegate\_helper.cc的類別Render\-View\-Host\-Delegate\-Helper所包含的Method(Create\-New\-Widget(int route\_id, WebKit::Web\-Popup\-Type popup\_type, Render\-Process\-Host* process))呼叫該介面,介面指的是Render\-Widget\-HostView::Create\-View\-For\-Widget\-(Render\-Widget\-Host* widget),因為CreateViewForWidget(RenderWidgetHost* widget)定義為static,在呼叫該Method時,程式設計師必須連同類別名稱、scope resolution operator(::)、Method名稱一起指定撰寫,請見圖~\ref{crossplatformcall}第八到十行。該原始碼檔案是一個與平台無關的程式碼,這個檔案是置放於chrome子專案中的browser目錄底下的tab\_contents目錄,請見圖~\ref{fig-folderstructure}。此目錄有置放其他與平台相關的程式碼,依照3.1節 - 檔案目錄結構的描述來進行解讀,因為類別Render\-View\-Host\-Delegate\-Helper在概念意義上,比較接近置放於tab\_contents目錄底下的其他類別,所以還是被置放在tab\_contents目錄底下。此檔案擁有與平台無關的特性,因此將它歸類成與平台無關的程式碼。此外本論文是以補抓某一時間點、快照式的觀點進行觀察,且Chromium專案是一個尚在持續發展的專案,當下與平台無關、相關的程式碼共同置放在同一目錄,但是將來順應架構設計上的改變,而有分離平台無關、相關
程式碼的計畫,針對這點我們並無法排除其可能性,必須長時間加以觀察。

以架構設計、檔案目錄結構構面論述上述觀察的結果,介面定義、實作、與平台無關的程式碼分別被置放%在不同的專案(目錄)中。
content\-/\textit{Source}\-/\textit{browser}\-/\textit{render\_host}\-,%目錄中包含browser控制render的定義,
chrome\-/\textit{Source}\-/\textit{browser}\-/\textit{render\_host}\-,%目錄中包含browser控制render的實作,
chrome\-/\textit{Source}\-/\textit{browser}\-/\textit{tab\_contents}\-目錄中,在上述目錄中分別包含browser控制render的定義、browser控制render的實作、browser將每一個tab中的內容交給render呈現的控制邏輯,在概念上上述三個目錄可分別視為一個模組,或是一個專案,且分別扮演Interface Project, Native Project, Cross-Platform Project的角色,所以上述三個專案(目錄)分別是Interface Project, Native Project, Cross-Platform Project三種樣式的已知案例,請見圖~\ref{chromium-project-category-analysis-view}。

以軟體建置構面論述上述觀察的結果,介面定義、實作、與平台無關的程式碼分別置放於content\-/\textit{Source}\-/\textit{browser}\-/\textit{render\_host}, chrome\-/\textit{Source}\-/\textit{browser}\-/\textit{render\_host}, chrome\-/\textit{Source}\-/\textit{browser}\-/\textit{tab\_contents}目錄中,Target:content\_browser涵蓋第一個目錄,Target:browser涵蓋最後兩個目錄。Target:content\_browser, Target:browser對應的子目錄名稱分別為content\-/\textit{Source}\-/\textit{browser}, chrome\-/\textit{Source}\-/\textit{browser},前者扮演Interface Project、後者扮演Native Project與Cross-Platform Project的角色,所以上述兩個專案(目錄)分別是Interface Project, Native Project與Cross-Platform Project三種樣式的已知案例,請見圖~\ref{chromium-project-category-analysis-view-deploymentview}。

這個實例已經顯現出Chromium專案為了分離與平台有關的實作,而定義出共同的介面,提供給與平台無關的程式碼參考。這與Project範疇樣式語言中Interface, Cross-Platform, Native Project樣式,連接形成的網絡,預設要解決的問題符合。

觀察Chromium專案社群文件後,我們發現該專案已經定義各種建置的target,利用Xcode即可進行編譯,並且產生對應不同target的安裝檔案。比如說,對Release target進行編譯建置,即可產出一個專屬於Mac OS X平台的安裝檔案,並且於Mac OS X平台上,針對該安裝檔案進行自動化驗收測試。所以chrome專案是一個Installation Project的已知案例,請見圖~\ref{chromium-project-category-analysis-view}。

在這個實例中,總共找到Interface Project, Cross-Platform Project, Native Project, Installation Project這四個樣式的已知案例。
\myGraphicBB{chromium-project-category-analysis-view}{Project 範疇樣式語言 - Chromium版本,以架構設計構面觀察。}{chromium-project-category-analysis-view.eps}
\myGraphicBB{chromium-project-category-analysis-view-deploymentview}{Project 範疇樣式語言 - Chromium版本,以軟體建置構面觀察。}{chromium-project-category-analysis-view-deploymentview.eps}

\subsection{探討SWT專案中Project 範疇樣式語言的已知案例}
首先從建置軟體的構面進行觀察,想要建置編譯SWT專案必須先取得兩個子專案的原始碼,這兩個子專案分別名為org.eclipse.swt與org\-.eclipse\-.swt\-.cocoa\-.macosx\-.x86\_64,前者簡稱為三明治專案,後者建稱為可可亞專案。前者是在所有SWT專案可以佈署的平台上,編譯SWT專案都必須取得的專案,後者是在Mac OS X上,編譯產出相容於64位元JVM的swt.jar時,才需要被取得。若是在Linux平台上,要產出相容於64位元JVM的swt.jar時,則需要取得org.eclipse.swt.gtk.linux.x86\_64專案的原始碼,並且在執行64位元JVM的Linux平台上進行編譯產出swt.jar。

因為SWT專案的建置引擎是Apache Ant\cite{ant},所以我們先找尋Ant script,並觀察其中關於建置的機制。Ant建置腳本檔案build.xml放在可可亞專案的目錄中,直接觀察SWT專案的FAQ文件指出,如果要編譯得到swt.jar必須在可可亞專案目錄底下,執行相容於Apache Ant規範格式的建置腳本build.xml。執行該腳本後,會在可可亞專案目錄底下得到編譯並封包完成的swt.jar。此外該腳本檔案會引用置放在三明治專案中的BuildFragment.xml檔案,實際觀察該檔案發現,大部分的編譯流程都是記載在該檔案中。大致步驟如下描述,將三明治專案底下,所有在Mac OS X平台下,必須要編譯的Java code,保留其目錄的層狀結構,一起複製到可可亞專案目錄底下的暫存目錄,當編譯與封包swt.jar檔的動作結束後,在可可亞專案目錄底下,即可發現swt.jar。依照上述流程,本論文推斷三明治、可可亞專案都是Installation Project的已知案例。

此外在可可亞專案目錄底下,存在有一些副檔名為jnilib的檔案,這些檔案隨著可可亞專案一起被取得,在JVM執行swt.jar時,必須由JVM載入這些動態連結檔案。字串A代表三明治專案版本控制系統中的最新版本編號,上述動態連結檔案名稱中必須包含字串A。這是一種管理動態連結檔案的機制,確保由最新版本的三明治專案所產出的swt.jar,會與對應該版本的動態連結檔案一同被佈署。這些動態連結檔案可以視為Shared Library,而可可亞專案是Single Shared Library Project的已知案例。

\begin{myTable}{subprojectfolderview}{org.eclipse.swt專案的子目錄}
\begin{tabular}[width=\textwidth]{|c|c|c|}
\hline
\multicolumn{3}{|c|}{專案名稱}\\
\hline
\multicolumn{3}{|c|}{org.eclipse.swt(三明治)}\\
\hline
子目錄&package/目錄名稱&檔案名稱\\
\hline
\textit{Eclipse SWT}&\textit{org.eclipse.swt.custom}&CTabFolder.java\\
\textit{Custom Widgets}&&\\
\textit{/common}&&\\
\hline
\hline
\textit{Eclipse SWT/common}&\textit{org.eclipse.swt.graphics}&Resource.java\\
\hline
\hline
\textit{Eclipse SWT/cocoa}&\textit{org.eclipse.swt.graphics}&GC.java\\
\hline
\hline
\textit{Eclipse SWT PI/cocoa}&\textit{org.eclipse.swt.internal}&NSObject.java,\\
&\textit{.cocoa}&OS.java,...\\
\hline
&\textit{library}&os.h, os.c, ...\\
\hline
\end{tabular}
\label{swtfolderview}
\end{myTable}

\begin{figure}
\linespread{0.8}
\begin{Verbatim}[numbers=left,framesep=1mm,numbersep=-12pt]
	節錄自CTabFolder.java
	public class CTabFolder{
		...
		void destroyItem(CTabItem item){
			...
			GC gc = new GC(this);
			...
			gc.dispose();
			...
		}
		...
	}
\end{Verbatim}
\caption{CTabFolder.java}
\label{ctabfolder}
\end{figure}

\begin{figure}
\linespread{0.8}
\begin{Verbatim}[numbers=left,framesep=1mm,numbersep=-12pt]
	節錄自Resource.java
	public abstract class Resource{
		...
		void destroy(){
		}
		//Template Method Pattern applied.
		//dispose() is a template method
		public void dispose(){
			...
			destroy();
			...
		}
		...
	}
\end{Verbatim}
\caption{Resource.java}
\label{resource}
\end{figure}

\begin{figure}
\linespread{0.8}
\begin{Verbatim}[numbers=left,framesep=1mm,numbersep=-12pt]
	節錄自GC.java
	//GC stands for Graphics Context
	public final class GC extends Resource{
		...
		//override destroy Method
		void destroy(){
			...
			data.fg.release();
			data.bg.release();
			...
		}
		...
	}
\end{Verbatim}
\caption{GC.java}
\label{gc}
\end{figure}

\myGraphicB{swt-design-view}{以相同介面對與平台相關之資源進行記憶體回收機制}{swt-design-view.eps}

\begin{figure}
\linespread{0.8}
\begin{Verbatim}[numbers=left,framesep=1mm,numbersep=-12pt]
	節錄自NSObject.java
	public class NSObject{
		...
		public void release(){
			OS.objc_msgSend(this.id, OS.sel_release);
		}
		...
	}
\end{Verbatim}
\caption{NSObject.java}
\label{nsobject}
\end{figure}

\begin{figure}
\linespread{0.8}
\begin{Verbatim}[numbers=left,framesep=1mm,numbersep=-12pt]
	節錄自OS.java
	public class OS extends C{
		...
		public static final native int objc_msgSend(int id, int sel);
		...
	}
\end{Verbatim}
\caption{OS.java}
\label{os}
\end{figure}

\begin{figure}
\linespread{0.8}
\begin{Verbatim}[numbers=left,framesep=1mm,numbersep=-12pt]
	節錄自os.c
	...
	jintLong rc = 0;
	...
	//對Apple Objective-C Runtime
	//呼叫objc_msgSend(arg0, arg1) 
	rc = (jintLong)((jintLong (*)(jintLong, jintLong))
	objc_msgSend)(arg0, arg1);
	...
	return rc;
	...
\end{Verbatim}
\caption{os.c}
\label{osc}
\end{figure}

接著,我們以檔案目錄結構、架構設計構面進行討論。首先必須針對Java語言程式碼檔案與類別之間的關係進行說明,若有一A.java原始碼檔案,該檔案代表A類別,於後續篇幅會將兩者混用。首先以檔案目錄結構構面輔助論述,請見表~\ref{swtfolderview}。針對特地為了Eclipse而實作且與平台無關的程式碼,SWT 專案開發者將上述程式碼集中置放於package org.eclipse.swt.custom\cite{swtbook},而子目錄Eclipse SWT Custom Widgets/common只涵蓋該package,再對類別CTabFolder進行觀察確定其為與平台無關之程式碼。因為在所有平台上建置SWT都需要涵蓋子目錄Eclipse SWT/common,所以該子目錄與平台無關。而僅在Mac OS X平台上建置SWT才需要涵蓋子目錄Eclipse SWT/cocoa,所以該子目錄與平台有關。上述兩個子目錄都涵蓋package org.eclipse.swt.graphics,類別Resource, GC分別被分類於Eclipse SWT/common/org.eclipse.swt.graphics, Eclipse SWT/cocoa/org.eclipse.swt.graphics,再分別對類別Resource, GC進行觀察確定前者與平台無關,而後者與平台有關。所以綜合以上現象,本論文推斷於概念上Eclipse SWT/cocoa/org.eclipse.swt.graphics為與平台相關之專案,Eclipse SWT/common/org.eclipse.swt.graphics為與平台無關之專案,Eclipse SWT Custom Widgets/common/org.eclipse.swt.custom為與平台無關之專案。

討論至此,我們尚無法確定Cross-Platform Project, Interface Project, Platform Independent Project之已知案例,因此必須再以架構設計構面進行討論,請見圖~\ref{ctabfolder}、圖~\ref{resource}、圖~\ref{gc}、圖~\ref{swt-design-view}。進一步觀察CTabFolder.java,於其中發現Method destroyItem(CTabItem item)呼叫gc.dispose(),接著觀察類別GC發現其與類別Resource之間存在繼承關係,為了回收各種與平台相關的資源所佔用之記憶體,且開發者希望僅相依一致性之方法,開發者提供抽象化類別Resource,用以封裝上述與平台相關資源之記憶體回收行為,於Resource類別的Method dispose()中,開發者定義資源回收之一致性方法,該Method呼叫Method destroy(),於與平台相關之子類別GC中,開發者override(覆載)Method destroy(),將回收資源之邏輯於Method destroy()中實作,上述設計為Template Method 之實踐\cite{gofdesignpattern}。根據上述類別Resource提供介面的功能,Eclipse SWT/common/org.eclipse.swt.graphics涵蓋類似類別Resource之其他類別,所以該子目錄可以視為Interface Project之已知案例,與平台無關的類別CTabFolder相依於類別Resource,並且利用類別Resource所提供的Method despose回收與平台相關之資源,Eclipse SWT Custom Widgets/common/org.eclipse.swt.custom涵蓋類似類別CTabFolder之其他類別,所以該子目錄可以視為Cross-Platform Project之已知案例,於類別GC中,開發者實作與平台相關之資源回收機制,Eclipse SWT/cocoa/org.eclipse.swt.graphics涵蓋類似類別GC之其他類別,所以該子目錄可以視為Native Project之已知案例。

此外於GC類別中,開發者實作Method destroy()並呼叫於類別NSObject中實作的Method release()。於NSObject類別中,開發者實作Method release()並呼叫於OS.java實作之static Method OS.objc\_msgSend(this.id, OS.sel\_release),Method objc\_msgSend(int id, int sel)是JNI之介面定義。於C程式語言原始檔案os.c中,開發者實作呼叫作業系統原生API的邏輯,請見圖~\ref{nsobject}、圖~\ref{os}、圖~\ref{osc}。
上述類別NSObject, OS與os.c皆為與平台相關之程式碼,且開發者於Eclipse SWT PI/cocoa/org.eclipse.swt.internal.cocoa, Eclipse SWT PI/cocoa/libray中置放與平台相關之程式碼,在兩者中分別存在以Java, C程式語言實作且與平台相關之程式碼,所以我們以Java-Native, C-Native Project定義與平台相關但是聚集Java, C程式語言程式碼之專案,上述兩專案即為實例,請見圖~\ref{swt-project-category}。%處於是Cross-Platform Project的已知案例。

%類別GC是與平台有關之程式碼
%類別Resource是一介面
%Interface Project,package org.eclipse.swt.graphics中共同存在介面和與平台相關之程式碼
%回頭關注類別CTabFolder,該類別相依類別Resource,並且具備與平台無關的特性,所以子目錄Eclipse SWT Custom %Widgets/common為Cross-Platform Project,package org.eclipse.swt.custom也可視為Cross-Platform Project。

%值得注意

在這個實例中,總共找到Interface Project, Cross-Platform Project, Native Project, Installation Project, Single Shared Library Project這五個樣式的已知案例。\myGraphicBB{swt-project-category}{Project 範疇樣式語言 - SWT版本,以檔案目錄結構與架構設計構面觀察。}{swt-project-category-analysis-view.eps}

%對於3.1節的結論如下。在兩個實例中,總共找到Interface Project, Cross-Platform Project, Native Project, Installation Project, Single Shared Library這五個樣式的已知案例。此外Chromium與SWT專案成員在開發跨平台軟體時,解決類似於Project 範疇樣式語言所描述的問題,從中得到的經驗與知識,形成了專屬於Chromium與SWT專案的樣式語言網絡關係,請見圖~\ref{chromium-project-category-analysis-view}、圖~\ref{chromium-project-category-analysis-view-deploymentview}、圖~\ref{swt-project-category}。

%\textit{“A language is a living language only when each person in society, or in the town, has his own version of this language.” }
%\begin{flushright}by Christopher Alexander, p.337,The timeless way of building(1979).\end{flushright}

%所以Chromium、SWT專案成員對於實踐持續整合,所累積的經驗與知識,形成了自己的樣式語言,雖然沒有完全符合Project 範疇樣式語言,卻實證了Project 範疇樣式語言是一個真實存在的樣式語言。

\section{探討Build 範疇樣式語言以Chromium專案為實例}

Local Private Build樣式的部份,Chromium專案規範其開發者,於簽入原始碼前必須執行的事項,
請見圖~\ref{chromiumprojectcommitworkflow}中關於Get your code ready小節的第一項Code的第一、二點與第二項\cite{chromiumcontributingcode}。開發者於程式碼撰寫完成後,在簽入程式碼前,必須先確定程式寫作的風格與專案定義的風格一致,並且測試要簽入版本控制系統的程式碼。Chromium專案團隊已經將Local Private Build定義於團隊的開發流程中。所以由上述內容可知Chromium採用Local Private Build。此外上述規範可以視為一種整合流程,因此Local Private Build樣式相依Integration Workflow樣式,所以Local Private Build樣式與Integration Workflow樣式之間有關聯產生。

\begin{figure}[h]
\fontsize{12pt}{10pt}\selectfont
\linespread{0.8}
\fbox{
\parbox{14.5cm}{
\begin{description}
\setlength{\parsep}{0ex minus5ex}
\setlength{\itemsep}{0ex minus5ex}
\setlength{\topsep}{0ex minus5ex}
\item Get your code ready
\begin{description}
\item 1. Code
\begin{description}
\item \textbf{1. must conform to the Chromium style guidelines.}
\item \textbf{2. must be tested, preferably with unit tests.}
\item 3. should be a reasonable size to review. Giant patches are unlikely to get reviewed quickly.
\end{description}
\item \textbf{2. Run the unit tests and the UI tests to make sure you haven't broken anything. You can ask someone to pass it to the try server.}
\item 3. ...
\end{description}
\end{description}
}
}
\caption{Chromium專案規範開發者於簽入程式碼前之執行步驟}
\label{chromiumprojectcommitworkflow}
\end{figure}

Remote Private Build樣式的部份,請見圖~\ref{chromiumprojectcommitworkflow}中關於Get your code ready小節,第二項中“You can ask someone to pass it to the try server.”,在簽入程式碼進入版本控制系統前,開發者必須將程式碼置入持續整合系統進行跨平台建置整合。若沒有問題發生,開發者才可以簽入程式碼。try server是由Buildbot持續整合系統所擔綱,負責將未簽入版本控制系統的程式碼,在所有要佈署的平台上執行建置整合工作,而Chromium專案團隊為了將此實踐比較無縫的融入流程中,已經開發一些工具幫助開發者實踐上述的步驟。此外Remote Private Build必須依賴持續整合系統所提供的功能,有支援此功能的持續整合系統有Team City, Pulse, ElectricCommander, Buildbot\cite{continuousdelivery} 。所以由上述內容可知Chromium採用Remote Private Build。此外由上述關於Remote Private Build樣式的描述中可以發現,Remote Private Build樣式相依於Cross-Platform Integration樣式,所以Remote Private Build樣式與Cross-Platform Integration樣式之間有關聯產生。

Cross-Platform Integration樣式需要持續整合系統支援將同一次的建置工作,分派到不同平台上執行建置整合\footnote{提供這個功能的持續整合系統,此網頁中\cite{cimetrix}的Distributed builds項目有列出比較。}。舉例來說,Buildbot持續整合系統的console view\cite{buildbotconsoleview},呈現其進行建置整合工作時,將建置整合工作分派到對應不同平台的Builders。Buildbot持續整合系統的確有支援跨平台建置的功能,而Chromium團隊也將跨平台整合的想法,利用Buildbot具體實作出來了。所以由上述內容可知Chromium專案是Cross-Platform Integration樣式的已知案例。此外必須在各個平台上自動執行的腳本中,定義建置整合工作流程,所以Cross-Platform Integration樣式相依於Integration Workflow樣式,因此Cross-Platform Integration樣式與Integration Workflow樣式之間有關聯產生。

Integration Workflow樣式的部份,請參閱Chromium專案中對於Buildbot waterfall view 的介紹\cite{buildbotwaterfallview},其中指出Chromium專案將Buildbot持續整合系統中的XP Test dbg(1) Builder,...,XP Test dbg(6) Builder與Vista Test dbg(1) Builder,...,Vista Test dbg(6) Builder設定為可被觸發,當Win Builder dbg建置完成後產生產品封包檔案,可被觸發的Builder將針對產品進行end to end的驗收測試。因為Chromium專案團隊將整個Chromium專案的產出物視為一個整體,並且針對該產出物作驗收測試,所以Chromium專案中的子專案之間的產物,並沒有觀察到這樣的整合流程。此流程的好處是節省編譯建置產出產品封包檔案的時間,因為被觸發執行驗收測試的Builder,其測試對象是同一份封包檔案,沒有必要因為要在不同平台環境下執行測試,而在同一類平台\footnote{都是泛Windows平台,如Windows XP, Windows Vista。}下,執行編譯建置產生同一份封包檔案。所以由上述內容可知Chromium專案是Integration Workflow樣式的已知案例。

對於3.3節的結論如下。在這個實例中,總共找到Local Private Build, Remote Private Build, Cross-Platform Integration, Integration Workflow這四個樣式的已知案例。此外Chromium專案成員對於避免Broken Build發生所累積的經驗與知識,以及對於持續整合工作的用心與堅持,令人印象深刻。

\section{Single Responsible Person持續整合狀態監控負責人員產生機制}

團隊中有一個成員,負責監控整合過程中是否有Broken Build發生,並不是指有一個成員其主要工作就是負責維護整合流程,而是團隊中成員必須輪流擔任這個角色。如果團隊中只有一個人,或者是一組人,負責維護整合流程,所有關於建置整合的知識只會在這些人身上保存著,這些知識是開發軟體的實戰經驗,書本並不會描述該如何做會最適合當下所身處的團隊,如果這組人因故無法工作,建置整合的工作將形成瓶頸,困擾著軟體開發團隊。所以本著知識應該廣泛在團隊中流傳的觀點,應該設計一項機制,讓建置整合機制可以廣泛在團隊中傳播。

微軟Excel團隊有一項關於建置整合的傳統\cite{joelonsoftware},凡是有人簽入了有問題的程式碼,造成Broken Build,這位成員就必須負責維護建置整合流程,直到這位成員發現有其他成員,造成Broken Build。如果跨平台軟體開發團隊實踐上述機制,首先,團隊中對於建置整合的知識可以比較廣泛流傳,這些重要的知識不是限縮於少數人的頭腦中。其次,因為監看與維護建置整合流程是相當耗費心力的,成員們會謹慎撰寫程式、作測試。簽入程式碼時,注意有無衝突發生,避免本身的疏忽造成建置整合錯誤,這將有助於軟體開發的流暢。無形中提昇了成員對於開發軟體的紀律。最後,開發者撰寫僅適用於特定平台的程式碼,而造成在其他平台上發生Broken Build,因為在跨平台軟體開發團隊中,上述情況經常會發生,所以藉此機會可以學習到其他平台上開發軟體的知識。

\section{研究方法討論}

利用本論文所提出之四種構面進行觀察後,表~\ref{studymethodvsresult}呈現何種樣式可以利用上述構面被尋找到。此表中如有一列中各個欄位均為空白的情況,於觀察實例後,觀察者利用上述四種構面都未觀察到該樣式之實踐,例如我們未觀察到Chromium專案對於Single Shared Library Project與Patch Project樣式之實踐。此外表中如有一列中有多個欄位呈現${\surd}$符號,則表示本論文以多個構面進行觀察後,該樣式之實踐才得以確定。

\begin{table}[h]
\fontsize{10pt}{8pt}\selectfont
	\setlength{\abovecaptionskip}{0pt}
	\setlength{\belowcaptionskip}{10pt}
	\begin{center}
	\caption{研究方法中各構面與尋獲樣式比對表}\label{studymethodvsresult}
\begin{tabular}[width=\textwidth]{|c|c|c|c|c|}
\hline
樣式\構面&軟體建置&檔案目錄結構&架構設計&專案社群文件\\
\hline
\multicolumn{5}{|c|}{以Chromium專案進行Project範疇樣式已知案例探討}\\
\hline
Installation Project&${\surd}$&&&${\surd}$\\
\hline
Single Shared Library Project&&&&\\
\hline
Interface Project&${\surd}$&${\surd}$&${\surd}$&${\surd}$\\
\hline
Cross-Platform Project&${\surd}$&${\surd}$&${\surd}$&${\surd}$\\
\hline
Native Project&${\surd}$&${\surd}$&${\surd}$&${\surd}$\\
\hline
Patch Project&&&&\\
\hline
\multicolumn{5}{|c|}{以SWT專案進行Project範疇樣式已知案例探討}\\
\hline
Installation Project&${\surd}$&&&${\surd}$\\
\hline
Single Shared Library Project&${\surd}$&&&${\surd}$\\
\hline
Interface Project&&${\surd}$&${\surd}$&${\surd}$\\
\hline
Cross-Platform Project&&${\surd}$&${\surd}$&${\surd}$\\
\hline
Native Project&&${\surd}$&${\surd}$&${\surd}$\\
\hline
Patch Project&&&&\\
\hline
\multicolumn{5}{|c|}{以Chromium專案進行Build範疇樣式已知案例探討}\\
\hline
Local Private Build&&&&${\surd}$\\
\hline
Remote Private Build&&&&${\surd}$\\
\hline
Cross-Platform Integration&&&&${\surd}$\\
\hline
Integration Workflow&&&&${\surd}$\\
\hline
\multicolumn{5}{|c|}{查閱書籍與網站進行Good Habit範疇樣式已知案例探討}\\
\hline
Single Responsible Person&&&&${\surd}$\\
\hline
\end{tabular}
\end{center}
\end{table}

就Project範疇樣式進行討論,在觀察該範疇樣式時,我們必須以檔案目錄結構、架構設計、專案社群文件構面觀察才得以確定Interface Project, Cross-Platform Project, Native Project樣式之實踐。其中專案社群文件構面扮演著關鍵角色,若未能得到該構面之輔助,本論文將無法確定於Chromium, SWT實例中上述三個樣式之實踐。

就Build範疇樣式進行討論,在觀察該範疇樣式時,我們依賴專案社群文件構面進行觀察Local Private Build, Remote Private Build, Cross-Platform Integration, Integration Workflow樣式之實踐。%本論文對於Chromium專案於維護其專案知識分享網站之用心深感敬佩與汗顏。

就Good Habit範疇樣式進行討論,在觀察該範疇樣式時,我們瀏覽各種探討軟體開發與持續整合相關議題之書籍與網站,於上述媒體中尋找Single Responsible Person樣式之已知案例。所以以專案社群文件構面進行觀察時,我們並不能僅將視野限縮於與實例專案有關之文件與網站,必須廣泛涉獵與軟體開發、持續整合相關的知識。
\chapter{演化跨平台軟體持續整合樣式語言}
\vfill

\textit{A living language must constantly be re-created in each person's mind.}

\textit{Just so with pattern languages.}

\textit{Then, as each person makes up his own language for himself, the language begins to be a living one.}

\begin{flushright}
Christopher Alexander, The Timeless Way of Building(1979), p.338$\sim$339.\end{flushright}
\vfill
首先從發表於Portland Pattern Repository的持續整合樣式中,本論文擷取與Build 範疇語言相關的樣式,並將其融入Build 範疇樣式語言中,進而演化該樣式語言。此外本論文歸納整理他人實踐持續整合的經驗,並形成兩個樣式Collective Continuous Integration Ownership與Feedbacks,再將這兩個樣式融入Good Habit樣式語言中,進而演化該樣式語言。

\section{演化Build 範疇樣式語言}
依照Build 範疇樣式語言所要表達的意義,為了避免發生Broken Build而導致無法實踐持續整合,跨平台軟體開發團隊必須依循一定的流程執行持續整合。但是在開發流程中依然存在其他因素,使得跨平台軟體開發團隊無法實踐持續整合。經過演化後,該樣式語言表達如何幫助跨平台軟體開發團隊永續實踐持續整合。

\subsection{介紹確定要加入Build 範疇樣式語言的樣式}
第一個是Commit Early and Often,軟體開發者使用版本控制系統,長久以來於其中累積的經驗形成該樣式。此樣式論述下述問題,軟體開發者長時間沒有簽入程式碼到版本控制系統,在簽入時與其他開發者的程式碼產生衝突,如果衝突處理不當,這將導致Broken Build。雖然這個樣式涉及Software Configuration Management的實踐,但是實踐該樣式確實可以避免Broken  Build的發生,而且Software Configuration Management的實踐與實踐持續整合之間具有極大關連。所以將此樣式納入Build 範疇樣式語言中。

第二個是Continuous Integration Relentless Testing ,對開發跨平台軟體而言,最大的挑戰是複雜且數量眾多的測試工作,所以自動化測試是無法避免且必要執行工作。於探討Chromium專案時,我們發現Chromium專案對於end to end的自動化驗收測試的實踐,上述內容即該樣式的已知案例。

第三個是Single Unified Build Script ,在開發跨平台軟體專案時,必須集合專精於不同平台的專家們一起開發。專家們慣用的工具無法保證在每個平台上都可以獲得,而且擁有一樣的功能與效能。所以必須在各個平台上,自動產出專屬於該平台的建置流程檔案。於各平台上公認最佳的開發環境,專家們利用此建置流程檔案編譯出執行檔案。在取得原始碼的同時,Chromium專案成員利用gyp工具\cite{gyp},自動產出適合於各個平台的建置流程檔案,上述內容即該樣式的已知案例。

\subsection{介紹演化後的Build 範疇樣式語言網絡}
屬於Build 範疇樣式語言的樣式,與要融入該樣式語言的樣式,必須先建立起關係,藉著上述關係,形成新的Build 範疇樣式語言。
\myGraphicB{build-category-revised}{演化過的Build 範疇樣式語言}{build-catgory-pattern-language-network-revised.eps}
首先,將焦點放在Commit Early and Often樣式身上,為了實踐此樣式定義的流程,當程式碼可以完整涵蓋一項功能\cite{scmpatterntasklevelcommit},開發者必須驗證該程式碼,待程式碼經過驗證通過後,再將程式碼簽入版本控制系統。這些驗證的流程必須依賴於Remote Private Build與Local Private Build所定義的流程,確保頻繁簽入的行為不會影響簽入程式碼的品質。再來,將焦點放在Continuous Integration Relentless Testing樣式身上,此樣式與Integration Workflow之間有關連存在,這表示Integration Workflow樣式必須相依Continuous Integration Relentless Testing樣式,實際上,我們可以舉第三章關於論述Chromium專案實踐Integration Workflow的內容為例子,Chromium專案為了在同一類平台下都進行自動化測試,且顧及必須將整體時間降到可以接受的範圍,而設計出一套整合工作流程,可以看出兩個樣式之間的關係。最後,將焦點放到Single Unified Build Script樣式身上,此樣式與Integration Workflow之間有關連存在,這表示Integration Workflow樣式必須相依Single Unified Build Script樣式。因為在不同平台上都需要進行整合流程,所以在不同平台上的建置腳本上,記錄各個平台上的整合流程,並為了便利在不同平台上整合流程之進行,利用自動產生腳本的機制產生建置腳本。我們經由上述內容推斷兩者之間關係已經建立。最後,三個樣式加入Build 範疇樣式語言後,所形成的樣式語言網絡如圖~\ref{build-category-revised}所示。
%換個角度思考,Remote Private Build與Local Private Build這兩個樣式,定義了開發者執行簽入程式碼之前執行的步驟,如果開發者不時常簽入完整對應一個功能的程式碼,代表開發者並不常執行Remote Private Build與Local Private Build所定義的流程,這兩個樣式也就失去效用。

\section{演化Good Habit 範疇樣式語言}
本論文以完整的樣式呈現格式介紹Collective Continuous Integration Ownership與Feedbacks樣式,並且演化Good Habit 範疇樣式語言。 
原本Good Habit 範疇樣式語言著重於Broken Build發生後的處理策略。演化該樣式語言後,該樣式語言表達跨平台軟體開發團隊需要以積極的態度面對持續整合。

\subsection{介紹樣式Collective Continuous Integration Ownership}
\begin{description}
\item Name:\\
Collective Continuous Integration Ownership
\item Context:\\
依照敏捷式軟體開發模型所描述,當開發者完成小部分的功能後,涵蓋該功能的程式碼必須通過編譯、測試過後,整合進入還在開發的系統中\footnote{事實上,這樣的概念並不限定於敏捷式軟體開發模型。於著名論文\textit{No Silver Bullet — Essence and Accidents of Software Engineering(1987)}中,Fred Brooks提出軟體是由小部份功能慢慢成長而成的概念\cite{nosilverbullet},此概念與敏捷式軟體開發模型利用持續整合發展軟體系統的想法相似。}。所以持續整合驅動著敏捷式軟體開發模型,當Broken Build發生時,所有的開發動作都會停頓,然而開發者經常未注意到Broken Build的發生。此外,對於持續整合該如何進行,開發團隊成員不能明確掌握該流程,或是團隊中只有少數才明白其中的知識與經驗,造成在團隊中相關知識的傳遞十分有限,開發工作因為Broken Build而停頓的風險增加。 
\item Problem:\\
軟體開發團隊如何看待持續整合?
\item Forces: 
\begin{enumerate}
\item 進行軟體開發時,必須維持持續整合的進行,不能停頓。
\item 當完成一次整合工作,軟體開發團隊無法即時得知專案執行持續整合的結果。
\item 於軟體開發團隊中,在專案成員之間,執行持續整合的知識與經驗必須廣泛流通。 
\end {enumerate}
\item Solution: 
\begin{description}
\item Feedback機制:持續整合流程中有問題發生時,利用各種不同的Feedback機制,比如說,簡訊、電子郵件、甚至是實體的互動裝置,用以提醒每一個開發人員儘快排除該問題。 
\item 專人負責監控持續整合狀態:於團隊成員中,以輪流的方式指派專人負責維護持續整合流程。
\end{description}
\item Resulting Context:
\begin{description} 
\item 當有持續整合流程問題發生時,團隊成員們利用其常用的溝通媒介得知持續整合流程狀態。如果開發成員都處同一空間工作,甚至可以在該空間中安裝代表持續整合流程狀態的metaphor,用以提醒每個成員需要於最短時間內找出問題的原因,並修正之。
\item 在團隊中實行輪流維護持續整合系統的機制,關於執行持續整合的知識與經驗可望在團隊中流傳。
\end{description} 
\item Related Patterns:
\begin{description}
\item Single Responsible Person 介紹開發人員輪流維護持續整合的作法。
\item Feedbacks 介紹提醒開發人員持續整合狀態的作法。
\end{description}
\item Known Used:\\
Steve McConnell介紹關於Microsoft Windows 2000與Office專案執行整合工作的故事\cite{codecomplete}。因為這兩個專案程式碼行數皆達百萬行,所以需要有負責將軟體由原始碼編譯為可執行檔的建置團隊。當可執行檔順利產出時,該可執行檔交付給測試團隊進行測試。於專案的最後階段,開發團隊必須24小時隨時待命,當建置團隊發現建置無法順利完成,或是測試團隊進行煙幕測試發現問題時,開發團隊馬上回到辦公室把問題修正。這不是鼓勵開發團隊日夜工作不休息,在開發階段初期,Microsoft Windows 2000與Office專案並不會執行這樣鐵血的制度。即使於專案的末期,時程壓力大的時候,軟體開發團隊必須正視執行持續整合流程時產生的問題,並即時解決該問題,這樣的作法讓開發進度不受阻礙。
\end{description}

\subsection{介紹樣式Feedbacks}
\begin{description}
\item Name:\\
Feedbacks 
\item Context:\\
在軟體開發團隊中,每一個開發者正在努力撰寫程式、測試軟體,在五分鐘前,有一個開發者簽入的程式碼,程式碼因為無法通過自動化驗收測試項目a-1,所以持續整合系統的狀態是處於有問題的狀態,代表正在開發的軟體中存在缺陷。所有的開發者還在進行手頭上的工作,甚至有人正想要簽入程式碼,沒有人發現程式碼無法通過測試。理論上,整個開發步調應該被暫停,除非自動化驗收測試通過,否則各個開發者必須優先修復程式碼,以通過自動化測試。當程式碼修復,且自動化驗收測試通過後,各個開發者才繼續其開發工作。
\item Problem:\\
如何通知開發者持續整合流程有事件發生? 
\item Forces: 
\begin{enumerate}
\item 每一個開發者期望得知持續整合流程的錯誤訊息。
\item 每一個開發者慣用的通訊媒介不同。 
\item 為了要幫助每一個開發者了解現階段健康狀態,必須以融入開發團隊環境的方式與媒介來呈現該狀態。 
\end{enumerate}

\item Solution: 
\begin{enumerate}
\item 當持續整合流程出現問題時,持續整合系統必須通知團隊成員\cite{cruisecontrolpublisher},於各個成員最常使用的通訊媒介中,成員們設定接收持續整合系統所送出的通知訊息。當持續整合流程出現問題時,持續整合軟體發出通知,開發者暫停開發流程,並儘快處理持續整合系統提示的問題。 
\item 利用一種metaphor表達持續整合流程的健康狀態,所有團隊成員一起定義健康狀態指標和其呈現方式。比如說,將指標設定成驗收測試通過比率必須達100\%,不達到指標代表健康狀態不佳。若持續整合流程的健康狀態不佳,metaphor以特定的方式呈現,用以提醒成員必須努力加強測試來提高健康狀態指標。 
\end{enumerate}

\item Resulting Context:
\begin{description}
\item 每個開發者可以在第一時間發現持續整合流程發生錯誤,並且儘快解決問題。 
\item 每個開發者了解健康狀態指標的變化,並且試著改善該狀態。
\end{description} 
\item Related Patterns:
\begin{description}
\item Collective Continuous Integration Ownership樣式的實踐必須依賴Feedbacks樣式\item Single Responsible Person樣式的實踐必須依賴Feedbacks樣式。
\end{description}
\item Known Used:\\
Kohsuke Kawaguchi是Jenkins持續整合軟體\footnote{因為Oracle入主Sun Microsystems後,宣稱其握有Hudson這個名字的商標權,而原作者、社群與Oracle談判未果,遂將Hudson專案改名為Jenkins\cite{jenkinsci}。}的主要開發者,Kohsuke利用有色的燈泡,當作是Jenkins專案建置整合狀態的metaphor\cite{orb},透過觀察該metaphor傳達的訊息,Kohsuke可以藉此得知執行持續整合流程的狀態。
\end{description}

\subsection{介紹演化後的Good Habit 範疇樣式語言網絡}
\myGraphicM{googhabit-pl}{演化過的Good Habit 範疇樣式語言}{goodhabit-pattern-language-network.eps}
Good Habit 樣式語言原本只涵蓋Single Responsible Person一個樣式,因為要融入兩個樣式於此樣式語言中,所以本論文必須先建立起Single Responsible Person樣式與Collective Continuous Integration Ownership樣式、Single Responsible Person樣式與Feedbacks樣式之間的關係。首先關注Single Responsible Person樣式與Collective Continuous Integration Ownership樣式之間的關係,於Single Responsible Person樣式中,該樣式定義產生監控持續整合狀態人員的規則。於開發團隊中,為了達成廣泛地流傳持續整合相關知識,必須藉由上述規則輔助達成上述目標,所以藉由Single Responsible Person樣式,Collective Continuous Integration Ownership樣式所描述的一部份問題已經被解決。其次,關注Single Responsible Person樣式與Feedbacks樣式之間的關係。於Feedbacks樣式中,該樣式定義的機制用以提醒團隊成員關注持續整合流程狀態。每當執行持續整合流程後,負責監看持續整合流程狀態的成員需要得知其狀態,所以上述機制解決該成員的需要。於實踐Single Responsible Person樣式時,Feedbacks樣式解決其中一部份的問題。最後,探討Collective Continuous Integration樣式與Feedbacks樣式之間的關係。於開發團隊中,對於執行持續整合流程的狀態,每個成員需要即時得知,所以該狀態必須以metaphor呈現,用以加強成員們對於該狀態認知。一部份由Collective Continuous Integration樣式所描述的問題經由Feedbacks樣式得以解決。由上述三個關係得知,Collective Continuous Integration Ownership, Single Responsible Person, Feedbacks這三個樣式彼此有網絡關係,形成一種在特定情境解決特定問題的樣式語言。請見圖~\ref{googhabit-pl}。
\chapter{結論}
本章節的第一小節描述本論文的貢獻,第二小節描述後續可能的研究方向。
\section{論文貢獻}
本論文貢獻如以下三點描述。
\begin{description}
\item[提出用以觀察跨平台軟體持續整合樣式已知案例之研究方法] 本論文提出以軟體建置、檔案目錄結構、架構設計、專案社群文件四個構面為途徑進行探討樣式已知案例之方法,實際利用該方法觀察確實觀察到樣式之已知案例。

\item[探討已知案例] 於實際案例中,針對跨平台軟體持續整合樣式語言中三大範疇樣式語言,本論文確實發現上述三大範疇樣式語言之已知案例,實證該樣式語言之有效性、真實性。請見本論文第三章。

\item[演化樣式語言] 於實際案例中,本論文歸納其開發團隊對於實踐持續整合的經驗與知識成為樣式,並且擷取他人已經發表關於持續整合的樣式。綜合上述兩者中所得知樣式,我們將其加入跨平台軟體持續整合樣式語言,藉此演化跨平台軟體持續整合樣式語言。請見本論文第四章。
\end{description}

\section{未來可能的研究方向}
未來可能的研究方向如以下四點描述。
\begin{description}
\item [樣式語言必須與時俱進] 如果其他實際跨平台軟體專案實踐持續整合的經驗,無法利用跨平台軟體持續整合樣式語言表達時,則必須進行調整或是演化該樣式語言的動作。

\item[探討其他類型之程式語言實作之已知案例] 利用直譯式程式語言撰寫的跨平台軟體,例如以Python, Ruby與Perl語言實作之跨平台軟體,是否也如同以C, C++與Java語言等編譯式程式語言撰寫的跨平台軟體一般,可以找到跨平台軟體持續整合樣式語言的已知案例。

\item[以貼近持續整合之觀點深入探討] 針對Project範疇樣式語言進行探討,本論文以架構設計的觀點進行討論。針對跨平台軟體開發團隊導入樣式後所帶來的優勢,我們並未以持續整合的觀點深入探討,後續工作需以貼近持續整合之觀點進行探討樣式之已知案例。

\item[針對Chromium專案進行實驗] 針對Chromium專案,Single Shared Library Project樣式描述子專案間可以參照其他子專案之產出而不需完全重新建置底層元件,我們將導入該樣式於Chromium專案中,期望可以縮短Chromium專案建置之總時間。針對上述概念,我們將繼續規劃相關實驗進行驗證。若進行實驗所得到正面之結果,我們可以向Chromium專案提出建議,訴說導入Single Shared Library樣式之優點與必須額外克服的缺點。
\end{description}
\vfill
\newpage
\addcontentsline{toc}{chapter}{參考文獻}

\bibliographystyle{unsrt}
\bibliography{reference}

\end{document}  
